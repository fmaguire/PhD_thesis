\graphicspath{{chapters/6.Chapter_4/figures}}

\chapter{Endosymbiont Metabolic Analysis}

What is the endosymbiont doing? - metabolic map

How does it differ between day and night? - differential expression

What does it need?

What is it secreting? (control against NC64A chlorella signal peptides - are we missing anything obvious in transcriptome endosymbiont bin)






UBLAST and cut-off for KO annotation, IPATH2.0 for maps  \citep{Wisecaver2014}




Also of interest is the activity of fatty acid and amino acid biosynthesis in the endosymbiont.  This is because
all well studied examples of primary of secondary plastids that have lost photosynthetic activity appear to maintain 
essential roles in these and other areas of host metabolism.
In all well-studied cases to date, the plastid itself is retained, as it is known to be the site of essential biochemical processes unrelated to photosynthesis, including fatty acid and amino acid biosynthesis (Waller and McFadden 2005; Barbrook et al. 2006; Mazumdar et al. 2006). \citep{Donaher2009}



Additionally the comparison of photosynthetic \textit{Symbodinium} within their cindarian hosts in both autotrophic and mixotrophic conditions
is of interest \citep{Xiang2015}


Genes indeitnfied in paramecium tetaurelia involve din autogamy, reciliation and exocytosis respectively \citep{Arnaiz2010}
Interestingly highly differentially expressed genes appear to have a lower rate of gene loss \citep{Arnaiz2010}




Diffeq - single cell - technical noise important - must be quantifeid so it can be incorporated into 

Low amount of RNA in single cell presents a major difficulity leading to unavoiable technicla noise \citep{Brennecke2013}

Depth


Unfortunately no ERCC because reasons
http://www.nature.com/nmeth/journal/v10/n11/extref/nmeth.2645-S2.pdf




mycosporina amino acids - shikimate pathway - protect agains UVR \citep{Sommaruga2009} (Shick and dunlap 2002)
chlorella made MAA found in cilliates sonntag2007 
Negative in p bursaria chlorella thoufgh summerer2009


PSI-BLAST is more sensitive to distance evolutionary relationships that standard blast

biocyc and kegg 

\section{Methods}

\subsection{Transporter identification}

Transporter proteins were specifically investigated using the same pipeline in
host and endosymbiont binned transcripts (from \ref{chap:sct}), the Chlorella NC64A
genome \citep{Blanc2010}, the Coccomyxa sp C-169 \citep{Blanc2012}, 
and transcripts from a re-assembly of \citet{Kodama2014c} \textit{P. bursaria} 
- \textit{Chlorella sp.} transcriptome.  It should be reiterated that
\citet{Kodama2014c} filtered algal transcripts from their assembly and analysis
as their focus was on the expression profile of the host with and without 
the endosymbiont.

This pipeline operated as follows:

Transporter Classification Database (TCDB) \citep{Saier2006,Saier2009,Saier2014}



\begin{enumerate} 
    \item TCDB fasta sequences are downloaded (12,911 sequences as of 2015/9/2) and queried using each set PSI-BLAST 
    \item All sequences with more than 4 TM domains as determined by TMHMM


\end{enumerate}

orthagog - reimplemntation of orthmocl in C
e=1e-5 
i=1.5


https://code.google.com/p/orthagogue/


