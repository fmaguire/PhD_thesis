\graphicspath{{chapters/7.Discussion/figures}}

\begin{savequote}[75mm]
    ``a typical symbiotic \textit{Chlorella} strain common to all \textit{P. bursaria}
    strains does not exist''
    \qauthor{- \citet{Reisser1988}}
\end{savequote}

\chapter{Conclusions and Recommendations}

The main objective of this research was to generate ``omic'' resources


The feasbility of MDA-based single cell transcriptomics in the analysis
of complex multi-member eukaryotic systems.








PROFOUND THOUGHTS

Endosymbiosis 
is widespread across the tree of life and
ecologically

Forms a key part of the evolution of the euks


Understanding endosymbiosis in all its form is
fundamental to answering both high-concept key questions 
pertaining to the evolution
of the eukaryotic cell as well as specific mechanistic
and utilitarian questions about using
endosymbiosis, bioengineering, shit.


Green algal phylogenetics and taxonomy is a messy field still 
undergoing a high degree of flux. 

MDA-based genomes are difficult and prone to contamination 

MDA-based transcriptomics 


Digital normalisation and error correction in general is a hugely:ta
important technique by which 




The analysis of complex messay 


\section{Discussion}


\section{}

\subsection{Why not further integration?}

Why is there not evidence of tighter integration in this system.

Firstly, the protein import systems considered necessary for extensive
EGT to start taking place are complicated
in the cases of secondary and tertiary endosymbioses
than basic plsatids due to the increased number
of membranes that may need to be traversed especially
for import directly to the secondary plastid from the
host \citep{Hirakawa2012}.


Secondly, the unusual nuclear dimorphism of the host \textit{P. bursaria}
may prove a barrier to the vast majority of EGT activity. 

For successful transfer to take place between host and endosymbiont it
would be necessary for the gene to transfer not just from the 
endosymbiont to the transcriptionally active host MAC but to the germline
MIC.  Even then integration into the MIC would have to occur in such a way
that it would be correctly spliced and duplicated during the conversion of the MIC
back to the MAC.
Compounding this with sexual reproduction further decreases the probability of
effective integration.

It should be noted that the prototypical hosts of the endosymbiotically ``promiscuous''
green algae - \textit{Chlorella}, \textit{Coccomyxa} and \textit{Micractinium} 
all display germline sequestration either through the aforementioned dimorphism
in \textit{P. bursaria} or via standard metazoan germlines in the case of
\textit{Hydra} and the kleptoplastic sacoglossan sea slugs. 

