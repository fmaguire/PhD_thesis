\graphicspath{{chapters/7.Discussion/figures}}

\begin{savequote}[75mm]
    ``a typical symbiotic \textit{Chlorella} strain common to all \textit{P. bursaria}
    strains does not exist''
    \qauthor{- \citet{Reisser1988}}
\end{savequote}

\chapter{Conclusions and Recommendations}

There were 3 main objectives to this research:
\begin{itemize}
    \item Assessing the utility
of \textit{P. bursaria} and its endosymbioses with green algae
as a model organism for the study of the evolution of endosymbiosis.
    \item Generation of ``omic'' resources to inform further analysis of this system
    \item Investigating the the utility and feasibility 
            of current MDA-based
            single cell genomic and transcriptomic sequencing technique in the 
            analysis of complex multimember single-celled eukaryotic systems.
\end{itemize}

%Endosymbiosis is widespread across the tree of life and is a fundamental
%process in the evolution of the eukaryotes.   Unfortunately,
%it is difficult to study the evolution of this process as host
%and endosymbiont have developed obligate metabolic co-dependence.
%This means hypotheses about the mechanism of the endosymbiosis
%cannot be tested easily.  

Review of literature established that \textit{Paramecium bursaria} and its
4 algal endosymbionts: \textit{Chlorella variabilis}, \textit{Chlorella vulgaris},
\textit{Coccomyxa} sp., 
and \textit{Micractinium reisseri} make theoretically good model
organisms for the study of this process due to the existence
of extensive.   This is due to them being believed to share:
\begin{itemize}
    \item a well-developed background literature.
    \item facultative endosymbioses allowing elimination of the endosymbiont
        and re-introduction experiments.
    \item the potential of a functional and tractable RNAi system for hypothesis
        testing via gene transcription knock down.
    \item easily culturable and diverse available cultures.
\end{itemize}
However, with the exception of one transcriptomic analysis \citep{Kodama2014} 
there have been no analyses of these systems using 
contemporary transcriptomic, metabolomic or genomic techniques. 


%and the \textit{C. variabilis}
%NC64A and \textit{Coccomyxa subellipsoidea}

An investigation of elimination of the endosymbiont in the \textit{P. bursaria}
-\textit{M. reisseri} culture revealed that this endosymbiosis
may represent an obligate system, at least on the part of the
\textit{P. bursaria} SW1 host.  Even though 3 elimination methods and
a range of treatment concentrations were attempted (to minimise the 
risk that the host death was related to a susceptibility to a 
specific treatment) all methods resulted in the same eventual host death.
Future work is required to establish whether \textit{M. reisseri}
algae are also obligate endosymbionts.  If this mutually obligate system is 
the case then metabolic co-dependence may have become fixed between
\textit{P. bursaria} and \textit{M. reisseri}.
This can be tested by testing whether an axenic culture of 
\textit{M. reisseri} can be established. 
This could be achieved
by exploiting robustness of the chitinous cell wall
of the algae relative to the \textit{Paramecium} membrane. Gentle agitation
would allow the lysis of the endosymbiont without the lysis of a significant
number of the endosymbionts.  The difficulty in this would be optimising
the culture conditions for \textit{M. reisseri} 
as these endosymbiont algae are known to be relatively fastidious \citep{Hoshina2009}.

As the Yad1g1N culture has previously been established as
a facultative endosymbiosis 
further work could use the Yad1g1N and CCAP1660/12 
transcriptomes created here to attempt to identify key differences 
between them.  Identifying these differences may pinpoint
the mechanism by which metabolic co-dependence becomes fixed in \textit{P. bursaria}
- green algal endosymbioses.

The ITS2 sequence analysis of the CCAP 1660/12,
and CCAP 1660/13 cultures revealed that these cultures likely contained 
the same \textit{M. reisseri} endosymbiont and not a \textit{Coccomyxa}
endosymbiont as described in the culture collection. 
The identity of the Yad1g1N endosymbiont
as \textit{C. variabilis} 1N was also confirmed.
A taxonomic analysis of the \textit{Paramecium} species
should also be conducted targeting the single MIC copy of the rDNA
to confirm the host identities in the primary cultures. 

This ITS2 analysis also established that 
the photobionts in the CCAP 1660/12, CCAP 1660/13
and NBDJ Yad1g1N cultures most likely form
clonal photobiont cultures within their host and none
of the cultures show evidence of multiple species of photobiont.
This suggest that clonal photobiont samples exist
in nature as samples such as the CCAP 1660/12 and 1660/13
cultures were sampled directly from the environment.  
As the Yad1g1N culture creation involved
the isolation and purification of the 1N endosymbiont, 
clearing of the Yad1g host and then their subsequent
reintroduction this suggests that the photobiont
undergo little divergence and remain largely
homogeneous within the host.

Unfortunately, the utility of the single cell metagenome to further
test endosymbiont clonality was limited due to 
the high level of bacterial contamination.
New tools and methods need to be developed to process and cluster
MDA genomic contigs in the absence of reference genomes
due to the unreliability of coverage as a feature.  The existence
of this type of method optimised for ``de novo'' assembled eukaryotic data
would greatly aid the analysis of complex interacting eukaryotic
systems using MDA based genomics.   One approach could be to utilise
blanket normalisation methods and base the variational inference
off the normalised coverage and compositional features however,
but using a blanket coverage threshold instead of a relative one
there is the potential to discard a significant portion of sequencing
data.

Therefore, the first result chapter already
shed doubt on some of the arguments supporting the utility of
\textit{Paramecium} in the study of endosymbiosis. Specifically,
it is not necessarily facultative and metabolic dependence if not
necessarily co-dependence has become fixed in at least one
species.  Additionally, due to this evidence of diversity
in the relationships combined with the taxonomic turmoil and previous
mislabelling means that the utility of the reference literature
is reduced. Any data from literature prior to the establishment
of molecular taxonomy in these species needs to be carefully
revisited and verified before it can be effectively used to
contextualise the ``omic'' analyses.


In-depth analysis of the optimisation of pre-processing,
filtering, assembly, and binning of single cell transcriptomics
revealed that it was possible to generate and assemble
single cell RNA-seq datasets of complex eukaryotic
systems. Previous work has shown the potential utility
of ``de novo'' SCT in eukaryotic micro-organisms \citep{Kolisko2014}
however, this work represents the first analysis
using SCT to investigate a non-axenic eukaryotic system.
Particularly, this is also the first analysis of a pair
of interacting eukaryotic partners using single cell methods
in the absence of reference genomes. 
This analysis identified that GC\% based pre-assembly read partitioning
is ineffectual for these datasets, but taxonomic screening
is highly necessary to minimise the levels of bacterial contamination.
This data also emphasises the utility of phylogenetically
informed transcript binning processes instead of relying
exclusively on naive top BLAST hit approaches. 
This work also determined that current recommended practices
 in bulk RNA-seq, such as digital normalisation and error correction,
 are still highly useful techniques in the analysis and assembly
 of SCT datasets.
Future work could consider the utility of phylogenetically aware
kernels (e.g. \citet{Vert2002}) in the classification
of transcript bins. Pre-assembly read partitioning
should be revisited and the benefit of incorporating
additional sequence feature such as composition
and coverage data investigated in this form a pre-processing.


An analysis of the endosymbiont metabolism via expressed 
transporters and secreted proteins revealed novel
aspects of amino acid usage by \textit{M. reissieri}
as well as the potential synthesis of complex saccharides
such as raffinose and arabinose within the PV lumen. 
Metabolomic data supporting these hypotheses
were also presented.  While the untargeted metabolomic profiling
does require further optimisation, particularly GC-QTOF analysis
of carbohydrate metabolism, these approaches are highly
useful to supplement transcriptomic data.
Further, targeted mass spectrometry is required to confirm
the differential abundances of raffinose, arabinose.
Additionally, the targeted amino acid analysis requires
redone due to a failure to fit calibration curves to the peaks
generated by the majority of the amino acids. 

Comparison of the active metabolic network in \textit{M. reisseri}
during endosymbiosis to that of \textit{C. variabilis} 1N 
and the total metabolic capacity of \textit{C. variabilis} NC64A
and \textit{Coccomyxa subellipsoidea} also revealed 
unique traits.  Specifically, \textit{M. reisseri} does not express
aspects of fatty acid degradation present in the other endosymbionts
as well as having distinct amino acid degradation pathways that are
congruent with the identified alternative amino acid usage in this species. 

The discovery of novel traits support the potential
 utility for single cell transcriptomics and bulk metabolomic
 analysis for identifying the underlying molecular function
 of a given endosymbiotic relationship.
 Unfortunately, it also further underlines the diversity
 and variability displayed between different \textit{P. bursaria}
 - green algal systems.  


 Finally, an analysis of RNAi in \textit{P. bursaria}
 revealed a potentially inactive/absent dsRNA induced RNAi system
 in \textit{P. bursaria} SW1 (CCAP 1660/12).  
 The common pattern of presence and absence of the
 previously identified components of the 
 RNAi pathways in both \textit{P. bursaria} transcriptomes
suggests that this system is likely to be inactive or missing in 
\textit{P. bursaria}. The most significantly missing factor is that
of the Pds1 gene that has been implicated in the uptake of RNA
from the digestive vacuole. As this has been discovered to occur
at low and natural levels in \textit{P. tetaurelia} during normal
feeding \citep{Carradec2015} the potentially deleterious 
presence of eukaryotic algal endosymbionts
may offer an explanation for the deactivation/absence of this system.

An ``in-silico'' study of the number of potential siRNA ``collisions''
between the active \textit{Paramecium} transcriptomes and other
eukaryotic transcriptomes (particularly those belonging to the endosymbionts) 
supported this hypothesis.  Relatively more collisions occur between
the host and eukaryotic transcriptomes than do versus bacterial ones.
The greater number of collisions increasing the chance of deleterious
cross-talk taking place and thus likely increasing the fitness
cost of maintaining this system in the presence of eukaryotic endosymbiont. 
Additional work needs done to assess the exact nature of these collisions,
particularly between host and endosymbiont. 



As there is evidence of both specialised adaptations in each
host-algal system, including potentially mutually obligate dependencies,
why is there a lack of evidence evidence of tighter 
integration in this system?  Specifically, the type of genomic
integration displayed in other endosymbionts such as EGT.

Firstly, the protein import systems considered necessary for extensive
EGT to start taking place are complicated
in the cases of secondary and tertiary endosymbioses
than basic plastids due to the increased number
of membranes that may need to be traversed especially
for import directly to the secondary plastid from the
host \citep{Hirakawa2012}.
Secondly, the unusual nuclear dimorphism of the host \textit{P. bursaria}
and codon usage may prove a barrier to the vast majority of EGT activity. 

For successful transfer to take place between host and endosymbiont it
would be necessary for the gene to transfer not just from the 
endosymbiont to the transcriptionally active host MAC but to the germline
MIC.  Even then integration into the MIC would have to occur in such a way
that it would be correctly spliced and duplicated during the conversion of the MIC
back to the MAC.
Compounding this with sexual reproduction further decreases the probability of
effective integration.
It is notable that the prototypical hosts of the endosymbiotically ``promiscuous''
green algae - \textit{Chlorella}, \textit{Coccomyxa} and \textit{Micractinium} 
all display germline sequestration either through the aforementioned dimorphism
in \textit{P. bursaria} or via standard metazoan germlines in the case of
\textit{Hydra} \citep{Kawaida2013} and the kleptoplastic sacoglossan sea slugs \citep{Yellowlees2008}.



Future work, could attempt to use the genomic contigs generated here
and/or further sequencing to attempt to pin-point examples of endosymbiont
genes being present in host contigs and vice versa. Then, due to the rate of 
chimeric contigs in MDA, PCR and sanger sequencing could be used to confirm
any putative EGTs.

Another interesting angle of investigation of these systems is
analysing what host and endosymbiont are not expressing during endosymbiosis.
This was partially investigated by \citep{Kodama2014} however, 
the partial genome and transcriptome could be used to further
investigate this question.  Specifically, all genes present in the assembly
could be identified and annotated using standard annotation pipelines
and then the transcriptomes surveyed for their presence.  Any gene
present in the genome that is not observed in the endosymbiosis
transcriptomes could shed further light on the function
and evolution of these systems.


Ultimately, this work has identified the diverse, complex and distinct set
of traits that different \textit{P. bursaria}-green algal endosymbioses
and demonstrated that, while still nascent, single cell methodologies
can be amenable to the analysis of complex multimember eukaryotic
systems that lack prior genomic references.  Unfortunately, both due to the
diversity of the systems discovered in the \textit{P. bursaria}-algal endosymbioses
and the inability to induce RNAi the initial 
utility of the \textit{Paramecium bursaria} as a general model for
the evolution of co-dependence is less than initially believed.  
However, future work in these systems using the ``omic'' resources generated in 
this thesis as a base dataset could help us understand how such mechanistic
endosymbiotic diversity is possible even in closely
related host and endosymbionts species. Understanding the answer to this
question would greatly improve our understanding of the evolution of endosymbiosis
and ultimately the eukaryotic cell.
