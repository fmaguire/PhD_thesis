% intro should describe and justify this cahpter
% 
\graphicspath{{chapters/3.Chapter_1/figures}}

\chapter{Endosymbiont Diversity}\label{chap:endo_diversity}

Publications arising as a product of this work \citep{Chambouvet2015} 

\section{Introduction}

\subsection{Endosymbiont taxonomics}

As briefly discussed in \ref{chap:intro} the taxonomic relations
of the green algae and in particular the Chlorellacea 



\begin{figure}[h]
    \includegraphics[width=\textwidth]{its2_schematic.pdf}
    \caption{Structure of Eukaryotic nuclear ribosomal DNA.
        rRNA genes exist in tandem repeats separated by nontranscribed spacers (NTS).
        pre-rRNA contain 
    Redrawn from \citep{}}
    \label{fig;its2_schematic]}
\end{figure}



Algal is a clusterfuck





The nuclear ribosomal internal transcribed spacer 2 (ITS2) is a well used


ITS2 has shown particular utility in the identification and separation
of closely related green algal species \citep{Buchheim2011}.


However, this type of barcoding approach to species identification is 
imperfect 

\subsection{Clonality of endosymbionts}

Single cell genomes


\subsection{Elimination of endosymbionts}

Clearing endosymbiont - paraquat \citep{Hosoya1995a}, dark \citep{Karakashian1963}, DCMU german paper \citep{Reisser1976},
X-ray \citep{Wichterman1948}, cyclohexamide \citep{Weis1984,Kodama2007}



Paraquate Hosoya 
Kodama
%Fortunately, to this end, the only other published 2nd generation sequencing analysis of \textit{P. bursaria} and a green algal,
%endosymbiont: Kodama \textit{et. al.} 2014 \citep{Kodama2014} partially addressed this issue.  This analysis investigated 
%the differential global metatranscriptome profile of \textit{P. bursaria} Yad1g strain with and without its \textit{Chlorella variabilis} 1N endosymbiont 
%\citep{Kodama2014}.   While, this is a different strain of both host and endosymbiont to the CCAP1660/12 strains (\textit{P. bursaria} and \textit{Micractinium reisseri}) 
%used in this thesis it offers a potential avenue to investigate these other components.
%An attempt was made to replicate this work using the CCAP1660/12 strains, unfortunately, elimination of the endosymbionts without death of the host
%didn't prove possible in these strains.  Despite using 
%
%(despite earlier publications to the contrary) by either maintaining the culture in the dark \citep{Siegel1960} (although some studies have thrown doubt on
%        how effective this method is at completely elimiating the photobiont \citep{Tanaka2002}) or treatment with various titrations of herbicides 
%            (e.g. paraquat \citep{Hosoya1995a}) or the protein synthesis inhibitor cyclohexamide \citep{weis1984effect}).\footnote{
%        There are naturally aposymbiotic strains of \textit{Paramecium bursaria} \citep{Tonooka2002a}}






\section{Aim}

In this chapter I will determine the exact algal endosymbiont strains present
in the 3 principal \textit{Paramecium bursaria} cultures used throughout
this thesis and their relationships relative to one another and to
other green algae.

I will also use single cell genomics to investigate whether the algal
endosymbiont present in the \textit{Paramecium bursaria-Micractinium reisseri}
CCAP 1660/12 strains form a clonal population. 

Finally, I will discuss the attempts to remove the endosymbiont in the 
\textit{Paramecium bursaria} CCAP 1660/12 strain from the host.




\section{Methods}

\subsection{Taxonomic Investigation}
    
\subsubsection{ITS2 Sequencing}

ITS2 sequences were amplified using ITS2-S2F and CHsp 


\textit{Paramecium bursaria} 1660/13 ``Coccomyxa'' 
1k-10k used ITS2-S2F and ITS4-R
1D-3D using ITS2-S2F and CHsp


\subsubsection{18S Sequence determination}

\subsubsection{Phylogenetics}

\subsection{Single Cell Genomics}

\subsubsection{DNA Extraction}

\subsubsection{Library Preparation}

\subsubsection{Illumina Sequencing}

\subsubsection{Read pre-processing}
Trimmomatic
\subsubsection{Assembly}
Spades
\subsubsection{Comparative Analysis}
Map reads from individual assemblies back




\subsection{Endosymbiont elimination}

\subsubsection{Paraquat}

\subsubsection{Cyclohexamide}

\subsubsection{Darkness}




\section{Results}

\subsection{ITS2 Phylogeny}



\subsection{18S Fragment Phylogeny}

All assembled contigs from final transcriptome assembly (see \ref{chap:})


\subsection{Assembly}
\subsubsection{Comparisons}
\subsection{Elimination}
\subsubsection{Paraquat}
\subsubsection{Cyclohexamide}
\subsubsection{Darkness}



\section{Discussion}

\subsection{Reliability of Culture Collection}

CCAP has proven shit at being correct in what is actually in their cultures. 



\subsection{Clonality of endosymbionts}



\subsection{Diversity of traits in endosymbionts}



\section{Conclusions}

Despite over 50 identified strains 


Is the endosymbiont clonal?
ITS2 - yes it is
18S - why are we getting these fragments but kind of yes

Is it the species we think it is?
Yes 18S/ITS2 among green algae

Species concepts  - ITS2 vs 18S vs whatever \citep{Boenigk2012}

file:///home/fin/Downloads/AiM20120300003_34647180.pdf
gt





Potentially, we are faced with the possibility that metabolic co-dependence has become fixed in 1660/12
relative to the YADGN1 strain of Kodama.


