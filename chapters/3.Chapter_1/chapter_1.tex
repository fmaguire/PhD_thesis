% intro should describe and justify this cahpter
% 
\graphicspath{{chapters/3.Chapter_1/figures}}

\chapter{Endosymbiont Diversity}\label{chap:endo_diversity}

\section{Introduction}

Algal is a clusterfuck

\section{Aim}

In this chapter I will determine the exact algal endosymbiont strains present
in the 3 principal \textit{Paramecium bursaria} cultures used throughout
this thesis and their relationships relative to one another and to
other green algae.

I will also use single cell genomics to investigate whether the algal
endosymbiont present in the \textit{Paramecium bursaria-Micractinium reisseri}
CCAP 1660/12 strains form a clonal population. 

\section{Methods}

\subsection{Taxonomic Investigation}
    
\subsubsection{ITS2 PCR}

\subsubsection{18S Sequence determination}

\subsubsection{Phylogenetics}

\subsection{Single Cell Genomics}

\subsubsection{DNA Extraction}
\subsubsection{Library Preparation}
\subsubsection{Illumina Sequencing}
\subsubsection{Read pre-processing}
\subsubsection{Assembly}
\subsubsection{Comparative Analysis}


\section{Results}

\subsection{ITS2 Phylogeny}

\subsection{18S Fragment Phylogeny}

\subsection{}

\section{Discussion}

\section{Conclusions}





Despite over 50 identified strains 


Is the endosymbiont clonal?
ITS2 - yes it is
18S - why are we getting these fragments but kind of yes

Is it the species we think it is?
Yes 18S/ITS2 among green algae

Species concepts  - ITS2 vs 18S vs whatever \citep{Boenigk2012}

file:///home/fin/Downloads/AiM20120300003_34647180.pdf
gt
