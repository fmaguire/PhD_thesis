% intro should describe and justify this cahpter
% 
\graphicspath{{chapters/3.Chapter_1/figures}}

\chapter{Endosymbiont Diversity}\label{chap:endo_diversity}

Publications arising as a product of this work \citep{Chambouvet2015} 

\section{Introduction}

As briefly discussed in \ref{chap:intro} the taxonomic relations
of the green algae and in particular the Chlorellacea 



\begin{figure}[h]
    \includegraphics[width=\textwidth]{its2_schematic.pdf}
    \caption{Structure of Eukaryotic nuclear ribosomal DNA.
        rRNA genes exist in tandem repeats separated by nontranscribed spacers (NTS).
        pre-rRNA contain 
    Redrawn from \citep{}}
    \label{fig;its2_schematic]}
\end{figure}



Algal is a clusterfuck





The nuclear ribosomal internal transcribed spacer 2 (ITS2) is a well used


ITS2 has shown particular utility in the identification and separation
of closely related green algal species \citep{Buchheim2011}.


However, this type of barcoding approach to species identification is 
imperfect 


\section{Aim}

In this chapter I will determine the exact algal endosymbiont strains present
in the 3 principal \textit{Paramecium bursaria} cultures used throughout
this thesis and their relationships relative to one another and to
other green algae.

I will also use single cell genomics to investigate whether the algal
endosymbiont present in the \textit{Paramecium bursaria-Micractinium reisseri}
CCAP 1660/12 strains form a clonal population. 

\section{Methods}

\subsection{Taxonomic Investigation}
    
\subsubsection{ITS2 Sequencing}

ITS2 sequences were amplified using ITS2-S2F and CHsp 



\textit{Paramecium bursaria} 1660/13 ``Coccomyxa'' 
1k-10k used ITS2-S2F and ITS4-R
1D-3D using ITS2-S2F and CHsp




\subsubsection{18S Sequence determination}

\subsubsection{Phylogenetics}

\subsection{Single Cell Genomics}

\subsubsection{DNA Extraction}
\subsubsection{Library Preparation}
\subsubsection{Illumina Sequencing}
\subsubsection{Read pre-processing}
\subsubsection{Assembly}
\subsubsection{Comparative Analysis}


\section{Results}

\subsection{ITS2 Phylogeny}



\subsection{18S Fragment Phylogeny}

All assembled contigs from final transcriptome assembly (see \ref{chap:}

\section{Discussion}

\subsection{Reliability of Culture Collection}

CCAP has proven shit at being correct in what is actually in their cultures. 


\section{Conclusions}

Despite over 50 identified strains 


Is the endosymbiont clonal?
ITS2 - yes it is
18S - why are we getting these fragments but kind of yes

Is it the species we think it is?
Yes 18S/ITS2 among green algae

Species concepts  - ITS2 vs 18S vs whatever \citep{Boenigk2012}

file:///home/fin/Downloads/AiM20120300003_34647180.pdf
gt
