\graphicspath{{chapters/7.Chapter_5/figures}}

\chapter{RNAi Analyses}


\section{Introduction}

\subsection{RNAi in \textit{Paramecium bursaria}}

Is there significant cross-talk between endosymbiont and host? - is this why RNAi doesn't seem to work

What of the required RNAi components from Marker are present in genome and transcriptome host bins

What is the phylogeny of these components compared to other Paramecium/ciliate species?

eDicer experiment results - what is the overlap with host orthologs of known yeast lethal genes

rnai as cross-kingdom communication \citep{Weiberg2015}



FIGURE 5. Dating of genome duplication events


%\subsubsection{Single Cell genomic DNA preparation}
%Cells were transferred from their respective \(10\mu l\) droplets of sterile water to a microncentrifuge tube.
%CTAB method adapted from \citep{Winnepenninckx1993}.  Cells were disrupted by vortexing for 5 minutes with
%\(748.5\mu l\) CTAB extraction buffer (at 37\celsius) and ceramic beads (Sigma, \(425-600\mu m\), acid-washed).
%The tube was then incubated for 50 minutes at 37\celsius, vortexed for 5 minutes, and incubated for a further 50
%minutes at 60\celsius. DNA was extracted 3 times using phenol, chloroform and isoamylalcohol at a 25:24:1 ratio (pH 8),
%washed with 70\% ethanol and re-suspended in \(2.5\mu l\) TE buffer (pH 8). This DNA then underwent 
%Multiple displacement-based whole genome amplification using the Repli-G Single Cell Kit before purification using a 
%QIAmp DNA mini kit before elution in \(100\mu l\) elution buffer.  



\section{Aims}


