\graphicspath{{chapters/7.Chapter_5/figures}}

\chapter{RNAi Analyses}


\section{Introduction}

RNAi is a highly useful experimental methodology through which
many biological hypotheses can be tested.
Specifically, 






\subsection{RNAi pathway in \textit{Paramecium} sp.}

Is there significant cross-talk between endosymbiont and host? - is this why RNAi doesn't seem to work

What of the required RNAi components from Marker are present in genome and transcriptome host bins

What is the phylogeny of these components compared to other Paramecium/ciliate species?

eDicer experiment results - what is the overlap with host orthologs of known yeast lethal genes

rnai as cross-kingdom communication \citep{Weiberg2015}



FIGURE 5. Dating of genome duplication events



\begin{figure}
    \includegraphics[]{}
    \caption{Redrawing of known \textit{Paramecium} RNAi components/pathway from the marker paper}
    \label{fig:rnai_components}
\end{figure}


\section{Aims}

Induce RNAi based knock-down using transformed bacterial vectors.

RNAi via microinjection

Identify required components for endogenous or exogenous RNAi in \textit{P. bursaria} CCAP 1660/12 from transcriptome and genome

Investigate the possiblity of "host" - "endosymbiont" collision as a reason for disabling.


\section{Methods}

\subsection{RNAi feeding experiments}

Methods modified from ParameciumDB 

\begin{table}
    \begin{tabular}[|c|c|c|c|c|]
        \hline
    \textbf{Gene} & \textbf{Function} & \textbf{RNAi phenotype in}      & Vector Design & Reference \\
                  &                   & \textbf{\textit{P. tetaurelia}} &               &           \\
        \hline
        \textit{epi2} & Epiplasmin & ``Monstrous'' cells  & 500bp via \textit{Pst}I and \textit{Hind}III & \citep{Damaj2009} \\
        NSF & Membrane fusion factor & Lethal & 500bp via \textit{Pst}I and \textit{Hind}III & \citep{Galvani2002} \\
        pTMB.422c & Binding protein & Lethal & 500bp via \textit{Pst}I and \textit{Hind}III & \citep{Nowack2011} \\
        \textit{bug22} & Basal body/ciliary protein & Slow swimming and death & 313bp via \textit{Xba}I and \textit{Hind}III & \citep{Laligne2010} \\
        BBS7 & Ciliary ion transport & Fewer, shorter ciliar & 486bp via \textit{Xho}I and \textit{Hind}III & \citep{Valentine2012} \\
        PGM & PGM endonuclease & Post-autogamous cells unable to resume normal growth & 500bp via \textit{Pst}I and \textit{Hind}III & \citep{Baudry2009} \\
        \hline
    \end{tabular}
    \caption{Details of RNAi vectors used.  All constructs were cloned into a L4440 vector and used an Ampicillin resistance market}
    \label{tab:rnai_vecs}
\end{table}

            


\subsection{RNAi microinjection}


\subsection{Phylogenetic analysis of RNAi pathway}


\subsection{Investigation of RNAi ``cross-talk''}


\section{Results}

\subsection{RNAi feeding experiments}




\subsection{RNAi required components}



\section{Discussion}

\subsection{Exogenous RNAi is non-functional in \textit{P. bursaria} CCAP 1660/12}



\subsection{Endogenous RNAi is methodologically difficult}


While RNAi by microinjection repeatedly failed there is a still a high possibility
that this is more related to the methodological difficulty of this technique rather than
necessarily any 
\cref{fig:microinjection_nucleus}

\begin{figure}
    \includegraphics[width=\textwidth]{microinjection_hard.pdf}
    \caption{}
    \label{fig:microinjection_nucleus}
\end{figure}






\section{Conclusions}
