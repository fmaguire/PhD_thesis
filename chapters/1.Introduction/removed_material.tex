%%\subsection{Endosymbiosis and Eukaryogenisis}
%
%%Recombination via transduction, conjugation and/or transformation as found in the bacteria and archaea is
%%qualitatively different from that of recombination by sex found in the eukaryotes. 
%%Typically, the former leads to the generation of a pan-genome whereas the latter promotes vertical inheritance. \citep{Ku2015}
%
%Every known eukaryotic cell harbours at least 1 endosymbiont in the form of the mitochondria (or a highly
%reduced mitochondria-related organelle (MRO)). 
%
%Furthermore, the chloroplast found across every major branch of the eukaryotic tree of life (eTOL)
%which facilitates the fundamental process of photosynthesis.
%
%
%
%%What is endosymbiosis?
%%What are organelles?
%%What do we know about the processes by which it has occurred?
%%What differentiates organelles and endosymbionts if anything?
%
%
%Separating organelle and endosymbiont - hard \citep{Keeling2008} Nowack paulinella dispatch
%
%Inherent difficulty of applying a schema of discrete caterogies onto a continuous non-linear distribution
%
%
%The most prevalent example of this interaction is that of the mitochodria and chloroplasts 
%within the eukaryotes.
%
%The most prevalent examples, the chloroplasts and mitochondria of photosynthetic eukaryotes and 
%eukaryotes in general respectively, have become obligately dependent on one another to the point where distinguishing 
%the two partners as distinct organisms becomes difficult.
%With such difficulty distinguishing between endosymbiotic partners it is no 
%surprise that the discovery and study of endosymbiosis is relatively recent in 
%biology. 
%
%
%Eukaryote big bang after acquisition of mitochdonria - LECA Pihlippe2000 Koonin2007 - exact time and place and role unknown
%Mitochdonria post FECA pre-LECA or Mitochdonria as FECA
%
%Mithcondira first hypothesis - Lane and Martin 2010 bioenergetic explanation
%
%
%Thus far, I have only discussed primary endosymbioses, secondary yada yada
%
%
%
%
%
%\subsection{History of Endosymbiosis}
%
%The development of a formal endosymbiotic theory for organellar origin was a process that
%lasted 84 years. From, a first tentative mention in a 19th century footnote\footnote{``Sollte es sich definitiv best\"atigen, dass die Plastiden in den 
%Eizellen nicht neu gebildet werden, so w\"urde ihre Beziehung zu dem sie 
%enthaltenden Organismus einigermaassen an eine Symbiose erinnern. M\"oglicherweise
%verdanken die gr\"unen Pflanzen wirklich einer Vereinigung eines farblosen Organismus
%mit einem von Chlorophyll gleichm\"assig tingirten ihren Ursprung\ldots''
%\citep{Schimper1883} or to approximately translate: ``Should it be confirmed
%that plastids are not formed \textit{de novo} in oocytes, their relationship with
%their host could somewhat be considered as a symbiosis.  Possibly green plants owe
%their origin to a union of a colourless organism and a chlorophyll tinged one.''
%\citep{Neuhauser2014}. Interestingly, this paper also coined the term ``chloroplast'' \citep{Sapp2002}} 
%where Andreas Schimper considered the potential origin of plants by a symbiotic union. 
%
%
%to theory of symbiogenisis of Constantin Mereschkowsky (Константи́н Серге́евич Мережко́вский)\footnote{As 
%    an aside Mereschkowsky personally strikes a disturbing, having
%    collaborated with the Czar's secret police reporting ``dangerous''
%    Jews and other ``traitors'' in academia, leading a right-wing nationalist
%    and anti-semitic organisation in Kazan, promoting the genocide of the Jews
%    (pre-empting the Nazi concentration camps by 13 years) as well as numerous suspected and 
%    confirmed acts of paedophilia (including the rape of 26 girls as young as 6 
%    (earning him the moniker as the ``Marquis de Sade of Kazan'')).  
%    He even authored a novel presenting a paedophilic eugenic-fascist
%    society as a utopia.  
%    \citep{Sapp2002}.}
%
%22 years later that plastids were once independent photosynthetic bacteria 
%(``\textit{Cyanophyceae}'' approximately equivalent to the contemporary Cyanobacteria) 
%supported by his and others earlier research on lichens\footnote{Interestingly, 
%    there were several early antropomorphic socio-political 
%    interpretations of symbiosis in lichens with metaphors ranging from ``communistic'' 
%    (Herbert Spencer), ``slavery'' (Simon Schwendener - the discoverer of lichens as
%    super-organisms), and ``consortia'' (Johannes Reinke) \citep{Sapp2002}.}
%, his work on diatom chloroplast structure, and the discovery of 
%photosynthetic ``infusoria'' \citep{Mereschkowsky1905,Martin1999a,Sapp2002}. 
%
%Ivan Wallin extended Mereschkowsky's theories to the idea of that the mitochondria
%of the eukaryotes likely originated as another cellular organelle \citep{Wallin1922} 
%on the basis of earlier work by Paul Portier attemping to independently
%culture mitochondria \citep{Sapp2002} (interestingly, 
%Mereschkowsky had strongly refuted this idea just a couple of years earlier as 
%an idea that would ruin his ``theory of symbiogenisis'' \citep{Sapp2002} but 
%had elaborately comitted suicide the year prior to Wallin's publication \citep{Sapp2002}).
%
%However, these theories never truly reached mainstream acceptance at the time
%partially due to the repeated failures to generate direct evidence that mitochondria
%and chloroplasts were indeed once free-living bacteria by means of separating them
%and culturing them independently from their host cells.  Wallin, Portier, and Faministyn
%among other all attempted this but while some success was had by Faministyn
%culturing green algae from amoeba, invertebrate hosts all 3 consistently failed
%to robustly separate culture either mitochondria or chloroplasts and subsequently
%faced moderate to extreme scorn academically \citep{Archibald2014}.
%Furthermore, as conceptual and experimental breakthroughs occurred at increasing
%rate throughout C20, symbiotic theory fell increasingly out of favour among
%the luminaries of the day e.g. Thomas Hunt Morgan and Edmund Beecher Wilson (the
%latter of which was directly challenged by Mereschkowsky in his landmark 1905
%paper \citep{Mereschkowsky1905,Martin1999a}, and who in return referred to Mereschkowsky's
%theories as ``an entertaining fantasy'' \citep{Wilson1928,Martin1999a}
%\citep{Archibald2014}. The loud denunciations of Darwinian theories of evolution
%by these researchers, especially front-runners Mereschkowksy and Portier as being
%insufficient to explain biological novelty relative to their own theories
%regarding the acquisition and inheritance of microbes \citep{Sapp2002} no doubt
%contributed to this falling out of favour in the face of ascendent neo-darwinian
%modern synthesis.
%
%
%It wasn't until 45 years later and the definitive work of Lynn Margulis \citep{Sagan1967} 
%developed in the light of cutting edge discoveries attributable to the nascent
%molecular revoltion of biology. Despite ignorance of the earlier work of these 
%``symbiogeneticists'' \citep{Archibald2014} Margulis presented 
%
%
%One of Margulis key innovations was a recognition of the need to expand the
%work being conducted in eukaryotic genetic beyond just that of the nucleus
%\citep{Archibald2012}.
%
%
%
%
%\subsection{Studying endosymbiosis}
%
%
%
%As Kwang Jeon demonstrated host-symbiont co-dependence \citep{Jeon1972} can become fixed in as little as 18 months (200 generations approximately)  \citep{Jeon1978}
%
%However, there are still many open questions in the process of endosymbiosis,
%why do the two major facultative endosymbiotic events in the evolution of the eukaryotes
%appear to have only occurred once? By what process does endosymbiosis occur?
%Why are secondary photosynthetic endosymbioses relatively common?
%Unfortunately, these questions are very difficult to answer as in the most successful
%and abundant examples metabolic co-dependence between host and endosymbiont has become
%fixed.  This limits the abilities of researchers to investigate the nature of the
%relationship between host and endosymbiont and thus attempt to answer questions
%such as the above. 
%
%However,  
%
%
%
%
%
%%%%DIFF STUDYING ENDOSYMBIOSIS
%% metabolic co-dependence has become fixed
%% Kwang Jeon experimental induction  in ameoba 1976-1980 
%% there is a need for mocrobial model systems to learn about endosymbioese as they occur so they can be manipulated
%% cotinuing a long tradition \citep{OMalley2015}
%% still difficult to say how generelazable to the past ARCHIBALDBOOK
%
%However, despite the importance of this evolutionary process there is a relative
%paucity of models in which researchers can dissect the mechanisms by which it occur
%in\-situ before metabolic co-dependence becomes fixed.
%
%
%
%
%
%
%
%
%
%
%
%
%
%
%While there was some earlier evidence as to the presence of nucleic acids in 
%plastids such as the work by Stocking and Gifford Jr., who demonstrated that
%radio-labelled thymidine was incorporated into the chloroplast of \textit{Spirogyra}
%\citep{Stocking1959}.
%The unequivocable identification of DNA within chloroplasts came via the 
%the cytochemical and electron microscopy investigation of \textit{Chlaymdomonas moewusii} 
%by Ris \& Plaut \citep{Ris1962} and the subsequent work by direct isolation of
%dsDNA from \textit{Chlorella ellipsoidea}, \textit{Chlamydomonas reinhardtii}, spinach
%and beet leaves by Chun \textit{et al.} \citep{Chun1963}. The role played by
%\textit{Chlorella} here, once again, places it firmly at the roots of endosymbiotic
%theory and research.
%
%
%
%
%
%
%Margulis' theory was ultimately validated by later work such as that presented in 
%``Origins of prokaryotes, eukaryotes, mitochondria, and chloroplasts'' by Margaret Dayhoff 
%
%that proved endosmybiosis \citep{Schwartz1976}
%
%
%
%
%
%
%
%
%
%
%%%%% TRANSITION INTO WHY PHOTOSYNTHETIC ENDOSYMBIOSIS IS SO IMPORTANT
%\subsection{Photosynthetic endosymbiosis}
%
%While the acquisition of the mitochondria <++> %when 
%is the de facto definitive event in the evolution of the eukaryotic cell, 
%it is the later acquistion of photosynthetic organelles (plastid) and the subsequent 
%diversification of plants and algae (archaeplastida) that has fundamentally shaped global ecology.
%
%Photosynthesis is the biological process by which energy-rich reduced carbon compounds 
%are synthesised from atmospheric \(CO_{2}\) using captured light energy. 
%It is a specialised form of phototrophy, in which an organism is capable of trapping
%and utilising light energy using photopigments such as chlorophyll, carotenoids, phycobilins
%and retinal, flavins and bilins. 
%Photosynthesis, releases oxygen as a by-product of using water a the terminal electron receptor 
%(however, some bacteria, e.g. purple non-sulphur bacteria and green sulphur bacteria, use alternative
%receptors and are thus said to conduct anoxygenic photosynthesis).
%
%There are two stages to photosynthesis, consisting of light-dependent and light-independent reactions respectively.
%c3, c4, CAM
%
%
%As a process, photosynthesis likely predates the eukaryotes with stromatolites microfossils 
%indicating the presence of photosynthetic cyanbobacteria like organisms at least 3.5gya.
%
%
%
%
%%two open qs in photosynth endo is which cyanobacteria is the plastid closest to, when did it happen
%
%
%
%
%
%
%
%
%
%This is due to photosynthetic organisms (eukaryote and bacteria) forming the basal 
%link in almost every major global ecosystem where they act by converting solar energy 
%directly into the net primary productivity upon which all other organisms 
%deride sustenance \citep{Reyes-Prieto2007}.
%
%Plantes and algae convert approximately 258bn tons of atmospheric \(CO_{2}\) into biomass
%annually via photosynthesis \citep{}<++>
%
%The basal position of the plastid bearing branches of the eukaryotic tree of life %GOOD CITATION OF ETOL NOT BALDAUF 2000-2003
%suggests the deep seated importance ... something \citep{Yoon2004}
%
%
%The plastid likely originates from a single ancient primary endosymbiotic event \(\approx 1.6\)gya \citep{Yoon2004}.
%This endosymbiosis hypothetically takes place between a photosynthetic bacteria (cyanobacteria) and
%an unknown heterotrophic eukaryote \citep{Reyes-Prieto2007}.
%
%
%Phylogenetic analyses using fossils, cross-calibration, and/or molecular clock estimates indicate 
%this took place between \(0.9-1.7gya\) \citep{Yoon2004,Parfrey2011,Shih2013,McFadden2014} and 
%suggest that this endosymbiont likely branches basal to cyanobacteria, 
%
%
%
%Almost all plastid-encoded genes and endosymbiont genes that have moved to the host's nucleus are 
%cyanobacterial in origin 
%
%
%However, a small but significant subset of genes related to storage polysaccharide metabolism have been 
%considered by some researchers to display a separate \textit{Chlamydia}-like evolutionary signature.
%This has been cited as evidence of the role of a 3rd party in the primary endosymbiosis which has subsequently
%been lost.  However, the lack of present chlamydiales that infect archaeplastida despite broad host-range throughout 
%the eukaryotes and re-analysis demonstrating a mosaic of origins for these proteins provides no compelling
%evidence for this "menage-a-trois" scenario \citep{Domman2015}.
%
%
%
%
%This is likely more than 60mya  NOWACK2008
%
%
%Kleptoplastidy in Sacoglossan sea slugs, longevity (months) suggests they are maintained, 
%but evidence for endosymbiotic gene transfer in this system is scant (Pierce says maybe2012), Bhatta2013 says no.
%
%
%There are 3 key stages to ``domestication'' of a photosynthetic endosymbiont: getting photosynthate from it,
%procuring genes to supply it with protein, regulating its division \citep{McFadden2014}.
%plastidic phosphate translocators (pPTs) occur in all know plastids therefore likely predate divergence so were
%likely present in the initial acquisition.  Always antiporters, quid pro quo etc  most important
%
%Plastids have 100-200 genes (\textit{Paulinella} has <++>) from thousands as free-living
%Transfer of genes to host has advantages: endosymbionts are clonal in host - no exchange therefore no purifying seleections
%Muller's ratchet - mutation accumulation.
%Genes in sexual host - diploidy, recomb and purifying.
%Minimise oxygen species damage.
%Why are 100-200 maintained?
%Frequent on reconstructiomn in tobbaoca HUANG2003
%
%"Limited transfer window" hypothesis : hosts with multipel endosymbionts have a higher chance of transfer than hosts 
%with one or a few cf. Chlamydomonas to Plants Lister2003 Smith2011. More symbionts = more donrs 
%Tranfer of DNA INTO plastids is very rare  
%
%Most EGT will be lost but ones that are transolacted into endosymbiot will be kepts TOC TIC N-terminal signlal: peps
%Lock-in dependence on host - selectid for .
%
%
%Finally division: plastids are vertically inherited - therefore they need to replicate before host division.
%
%Too quick they kill host, too slow they don't work well and won't be abundant enough to go in daughter cells
%
%
%All divide using FtsZ apart from apicomplexan non-active ones Francia2012
%Most pllstid dividion machinery originates from homologous binary fission machinery in cyanobacteria
%but some key parts are eukaryotic - suggesting host is controlling 
%
%FtsZ asssembles into Z-ring at plastid equator using Min D,E motor proteins 
%This is tethered to inner env membrane with ARC6 (likely euk)
%Bunch of crap is recruited, another ring DRP5B (euk origin)- ring also forms on host side of outer plastid membrane (PD)
%\citep{McFadden2014}
%
%
%
%
%
%
%
%
%
%Subsequently, there appear to have been multiple secondary endosymbioses between plastid-bearing eukaryotes
%and other eukaryotes.
%
%
%
%
%
%
%Plastids-bearing eukaryotes 
%
%
%
%
%
%
%Photosynthetic endosymbiosis has occurred 
%
%
%
%
%1.558gya 
%red-green algae split 1.5gya
%1.3gya origin second endosymbiosis \citep{Yoon2004}
%
%
%
%
%
%%STATE OF PHOTOSYNTHETIC ENDOSYMBIOSIS
%
%% Primary likely occurred once but all over the place there are secondard
%%endosymbioses
%
%%% With the exception of the primary endosymbioses of the archaeplastida, almost all oxygenic phototrophs are believed to have arisen via secondary or higher order symbioses \citep{Hoshina2009} {}<++>
%
%%plastids all over the place even (INCLUDE OTHERS) rappemonads \citep{Kim2011a}
%
%
%
%
%
%
%
%%%Importance of gene duplication in explaining fungal metabolic diversity compared to
%%%HGT however, this is specifically related to gene clusters both GD and HGT more
%%%pronounced in clustersA
%%%GD dominant and across all taxa, HGT lineage specific innovation
%%%The disproportionate effect of GD and HGT on clustered genes renders metabolic gene clusters into hotspots of metabolic innovation and diversification in fungi
%%%Intriguingly, earlier diverged fungi had lower numbers of duplicated EC-annotated metabolic genes per genome possibly spurious due to few genomes
%%%\begin{math} 2.8\% \end{math}
%%%\citep{Wisecaver2014}, different from animal 
%%%HOX tandem duplication clusters because they are evolutionarily unrelated
%%%but can't tell if there 
%%%are clusters in \textit{Paramecium}
%%%In archaea and bacteria ``extensive gene loss and horizontal gene transfer leading to innovation are the two dominant evolutionary processes, and yields robust estimates of the supergenome size.''
%%%(note that this paper looks at 34 bacterial groups but only 1 archaeal) \citep{Puigbo2014}
%%%Recent emergence of ubiquitous bacterial horizontal regulatory transfer 
%%%``bacterial genes can rapidly shift between multiple regulatory modes by acquiring functionally divergent nonhomologous promoter regions'' 
%%%Same forces that drive coding HGT can also transfer regulatory non-coding regions that can have profound phenotypic consequences \citep{Oren2014}
%%%``the ubiquity and extent of HRT have not been appreciated before the study of Oren et al'' \citep{Koonin2014}
%
%%%Eukaryote genome biology is important but very skewed \citep{DelCampo2014}
%%%\textit{Paramecium bursaria} 
%
%
%%%%%``There are numerous features that are specific for eukaryotes and can be traced back to the last 
%%%Eukaryotic common ancestor (LECA), such as the nucleus, the endomembrane system62–64, the 
%%%Mitochondrion65,66, spliceosomal introns67,68, linear chromosomes with telomeres synthesized by 
%%%Telomerases69, meiotic sex70, sterol synthesis71, unique cytokinesis structures72 and the capacity 
%%%For phagocytosis'' \citep{Gribaldo2010}
%
%
%
%Acquisition of phototrophy does not commit an organism to a phototrophic lifestyle
%as can be observed in various the transitions of free-living autotrophic algae 
%to obligate parasites.  For example, apicomplexans such as \textit{Plasmodium} 
%(causative agent for malaria) are derived from red-algae (secondary plastid).
%There are also examples of green algae such as \textit{Helicosporidia} that 
%adopt a parasitic lifestyle despite having primary plastids. 
%Intriguinly, H. parasiticum doesn't seem to demonstrate the same level of
%genome reduction as other paraistes.\citep{Pombert2014}
%``within this single lineage are found free-living autotrophs like most
%other green algae, but also a variety of symbiotic species,
%opportunistic pathogens, and perhaps even obligate intracellular
%parasites,''
%``but a variety or parasitic lineages
%had at one time photosynthetic ancestors, including oomycetes,
%several dinoflagellates, and most famously the apicomplexan
%parasites such as the malaria parasite, Plasmodium (refs 10-11)
%and references therein)''
%\citep{Pombert2014}
%Plasmodium switched to parasitism over 1bya by some estimates, another example
%of difficult to discern processes because all traces have been lost (like symb)
%
%
%
%``Molecular clock analyses have esti- mated the origin of the green lineage between 700 and 1500 mya (Douzery et al., 2004; Hedges et al., 2004; Berney and Pawlowski, 2006; Roger and Hug, 2006; Herron et al., 2009)''
%\citep{Leliaert2012}
%``chlorophyte-streptophyte split at 700-1500Mya with similar refs to leliaert'' \citep{DeWever2009}
%Trebouxio somewhere between 500-1000Mya, \citep{DeWever2009}
%
%
%
%Says \citep{DeWever2009} indicates chlorellaes 100mya \citep{Pombert2014}
%
%Helicosporidia has nearly all metabolic genes of chrloella and coccomyxa apart from
%a few minor components of photosynthesis \citep{Pombert2014} from genome seq(all genes relating to light harvesting andelectron transport are missing)
%
%
%
%
%srrison1945equenced chlorophytes range from  67 64 65 64 59 60 gc percentage \citep{Blanc2010a}
%
%sequenced paramecium genomes ranges from 28.2 25.8 28.0 24.1 gc percentage \citep{McGrath2014}
%




