% The Thesis Introduction
% Lit review, rationale, permission from copyright holder for figs if unchanged, all should be modified, should close with statement of aims to be addressed
\graphicspath{{chapters/1.Introduction/figures/}}

\begin{savequote}[75mm]
Hofstadter's Law: It always takes longer than you expect, even when you take into account Hofstadter's law
\qauthor{- Douglas Hofstadter: \textit{G\"odel, Escher, Bach: An Eternal Golden Braid, 1979}}
\end{savequote}

\chapter{Introduction}

\begin{itemize}
    \item Micractinium - history, biology, knowns, importance
    \item Paramecium bursaria - history, biology, importance
    \item Endosymbiosis - general, evolution of, SET/ratchet gene flow
    \item Endosymbiosis between Para and Micractinium, knowns and why interesting?
\end{itemize}

DUE DATE: END OF JANUARY

\section{Endosymbiosis}

% What is endosymbiosis?
% Why is it important to understand?
% What does PbMr offer to this?
% How can I use them to help?

Endosymbiosis has shaped/driven the evolution of eukaryotic cell.  %% shaped global climate, allowed evolution of metazoa {}<++>%%
However, despite the importance of this process in shaping the world as we know
it there is a relatively paucity of models in which to attempt to dissect the 
mechanisms by which it can occur. %% EVIDENCE {}<++>%%
%% With the exception of the primary endosymbioses of the archaeplastida, almost all oxygenic phototrophs are believed to have arisen via secondary or higher order symbioses \citep{Hoshina2009} {}<++>
Since its first tentative discussion in a 19th century footnote
\footnote{``Sollte es sich definitiv best\"atigen, dass die Plastiden in den 
Eizellen nicht neu gebildet werden, so w\"urde ihre Beziehung zu dem sie 
enthaltenden Organismus einigermaassen an eine Symbiose erinnern. M\"oglicherweise
verdanken die gr\"unen Pflanzen wirklich einer Vereinigung eines farblosen Organismus
mit einem von Chlorophyll gleichm\"assig tingirten ihren Ursprung\ldots'' \citep{Schimper1883}} 
to the ground-breaking work of Lynn Margulis \citep{Sagan1967} 


Margulis recognised a need to expand investigation of eukaryote genetics beyond 
just the nucleus \citep{Archibald2012}


The ciliate \textit{Paramecium bursaria} and green algae \textit{Micractinium reisseri}




% QUOTE FROM SOURCE
%          The endosymbiotic  theory was first proposed  as a hypothesis in 1883 by a botanist  Andreas Schimper. He suggested that plastids  are descendants  of cyanobacteria  which were introduced into the cell by the process of phagocytosis (Heldt & Heldt,2005). The plastid developed a symbiotic relationship with the cell and eventually it became so  dependent on the cell that “divorce “ was no longer a viable option and they became “committed “ to each other.
%               In 1905 Konstantin Mereschowsky , a Russian botanist , proposed that chloroplasts were at one time prokaryotes that  lived independently of the eukaryotic cells. Chloroplasts  however  “married” into the eukaryotic cells and they have been inseparable  ever since (Mereschowsky Tree of Life,2001). The same relationship was proposed by Ivan Wallin, an American biologist, in the 1920s with respect to the mitochondria. The mitochondria is another organelle that was suggested as being introduced into eukaryotic cells and remaining permanently in the cells.However Wallin's theory on the mitochondria was criticized harshly and it was eventually discarded by Wallin himself ( What is Endosymbiosis,2006)
%







Importance of gene duplication in explaining fungal metabolic diversity compared to
HGT however, this is specifically related to gene clusters both GD and HGT more
pronounced in clustersA
GD dominant and across all taxa, HGT lineage specific innovation
The disproportionate effect of GD and HGT on clustered genes renders metabolic gene clusters into hotspots of metabolic innovation and diversification in fungi
Intriguingly, earlier diverged fungi had lower numbers of duplicated EC-annotated metabolic genes per genome possibly spurious due to few genomes
\begin{math} 2.8\% \end{math}
\citep{Wisecaver2014}, different from animal 
HOX tandem duplication clusters because they are evolutionarily unrelated
but can't tell if there 
are clusters in \textit{Paramecium}




in archaea and bacteria ``extensive gene loss and horizontal gene transfer leading to innovation are the two dominant evolutionary processes, and yields robust estimates of the supergenome size.''
(note that this paper looks at 34 bacterial groups but only 1 archaeal) \citep{Puigbo2014}
Recent emergence of ubiquitous bacterial horizontal regulatory transfer 
``bacterial genes can rapidly shift between multiple regulatory modes by acquiring functionally divergent nonhomologous promoter regions'' 
Same forces that drive coding HGT can also transfer regulatory non-coding regions that can have profound phenotypic consequences \citep{Oren2014}
``the ubiquity and extent of HRT have not been appreciated before the study of Oren et al'' \citep{Koonin2014}



Eukaryote genome biology is important but very skewed \citep{DelCampo2014}
\textit{Paramecium bursaria} 


``There are numerous features that are specific for eukaryotes and can be traced back to the last 
eukaryotic common ancestor (LECA), such as the nucleus, the endomembrane system62–64, the 
mitochondrion65,66, spliceosomal introns67,68, linear chromosomes with telomeres synthesized by 
telomerases69, meiotic sex70, sterol synthesis71, unique cytokinesis structures72 and the capacity 
for phagocytosis'' \citep{Gribaldo2010}



Margulis was great but it wasn't until the work of ``Origins of prokaryotes, eukaryotes, mitochondria, and chloroplasts.'' by Dayhoff that proved endosmybiosis \citep{Schwartz1976}




endosymbiosis is not unique to the eukaryotes (with at least one example in the
bacteria \citep{vonDohlen2001} and one symbiosis (not endo) in the Archaea \citep{Huber2002})


Other archaea symbioses:
Ignicoccus hospitalis/Nanaoarchaeum equitans, Cell-cell contact, Transfer of essential biological macromolecules from host to symbiont
SM1/Thiothrix   Cell appendages Sulfur redox cycling?
ANME-Archaea/sulfate reducing Bacteria  Cell-cell contact/ no direct contact    Anaerobic methane oxidation coupled to sulfate reduction
Methanogenic Archaea/Ciliates, Archamoebae   Endosymbiotic   Methanogen/hydrogenosome association, transfer of hydrogen, and/or C1/C2 compounds
Porifera-associated Thaumarchaeota  Tissue associated   Ammonia oxidation
Diverse associations between insect and vertebrate guts
\citep{Wrede2012}




While there was some earlier evidence as to the presence of nucleic acids in 
plastids such as the work by Stocking and Gifford Jr., who demonstrated that
radio-labelled thymidine was incorporated into the chloroplast of \textit{Spirogyra}
\citep{Stocking1959}.
The unequivocable identification of DNA within chloroplasts came via the 
the cytochemical and electron microscopy investigation of \textit{Chlaymdomonas moewusii} 
by Ris \& Plaut \citep{Ris1962} and the subsequent work by direct isolation of
dsDNA from \textit{Chlorella ellipsoidea}, \textit{Chlamydomonas reinhardtii}, spinach
and beet leaves by Chun \textit{et al.} \citep{Chun1963}. The role played by
\textit{Chlorella} here, once again, places it firmly at the roots of endosymbiotic
theory and research.



Acquisition of phototrophy does not commit an organism to a phototrophic lifestyle
as can be observed in various the transitions of free-living autotrophic algae 
to obligate parasites.  For example, apicomplexans such as \textit{Plasmodium} 
(causative agent for malaria) are derived from red-algae (secondary plastid).
There are also examples of green algae such as \textit{Helicosporidia} that 
adopt a parasitic lifestyle despite having primary plastids. 
Intriguinly, H. parasiticum doesn't seem to demonstrate the same level of
genome reduction as other paraistes.\citep{Pombert2014}
``within this single lineage are found free-living autotrophs like most
other green algae, but also a variety of symbiotic species,
opportunistic pathogens, and perhaps even obligate intracellular
parasites,''
``but a variety or parasitic lineages
had at one time photosynthetic ancestors, including oomycetes,
several dinoflagellates, and most famously the apicomplexan
parasites such as the malaria parasite, Plasmodium (refs 10-11)
and references therein)''
\citep{Pombert2014}
Plasmodium switched to parasitism over 1bya by some estimates, another example
of difficult to discern processes because all traces have been lost (like symb)



``Molecular clock analyses have esti- mated the origin of the green lineage between 700 and 1500 mya (Douzery et al., 2004; Hedges et al., 2004; Berney and Pawlowski, 2006; Roger and Hug, 2006; Herron et al., 2009)''
\citep{Leliaert2012}
``chlorophyte-streptophyte split at 700-1500Mya with similar refs to leliaert'' \citep{DeWever2009}
Trebouxio somewhere between 500-1000Mya, \citep{DeWever2009}



Says \citep{DeWever2009} indicates chlorellaes 100mya \citep{Pombert2014}

Helicosporidia has nearly all metabolic genes of chrloella and coccomyxa apart from
a few minor components of photosynthesis \citep{Pombert2014} from genome seq(all genes relating to light harvesting andelectron transport are missing)




sequenced chlorophytes range from  67 64 65 64 59 60 gc percentage \citep{Blanc2010a}

sequenced paramecium genomes ranges from 28.2 25.8 28.0 24.1 gc percentage \citep{McGrath2014}

\section{\textit{Paramecium bursaria}}

\citep{Corliss1974} 





Indeed, when one considers the influence Tracey Sonneborn's discovery of non-mendelian
inheritance non-mendelian inheritance in the related \textit{Paramecium (aurelia)}, where he showed
cytoplasmic inheritance of features such as cilia orientation \footnote{``Pieces of cortex of Paramecium can be grafted onto a whole cell and
    become integrated, yielding a modified cortical pattern which is maintained through both sexual and asexual reproduction.'' \citep{Beisson1965}}

the development of Margulis' forumaltion of endosymbiosis \citep{Margulis1998} it is
quite appropriate that we are revisiting this organism

Sonneborn's classic review \citep{Sonneborn1950} proved a verdant source of methodologies
for research on  \textit{P. aurelia} and \textit{Paramecium} in general

\begin{figure}[h!]
    \caption{\textbf{A}: Carving of Christiaan Huygens (1629-1695), the prominent Dutch Golden Age mathematician and scientist and contemporary of Antoni van Leeuwenhoek, from a medallion by Jean-Jacques Cl\'erion 1679 (reproduced from \citep{Huygens}). \textbf{B}: Likely the first sketch of the micro-organism that we now know as \textit{Paramecium} by Christiaan Huygens in a letter (No. 2133, 11th of August 1678) to his father Constantijn Huygens. An approximate translation of the accompanying text goes as follows ``I have twice seen in this water an animal 10 times as large as the others and with feet all over its body and a narrow form. 4 or 5 feet stirred even when the animal was at rest. It moves as fast as the others, turning and spinning in the water. Hartfoecker thinks he may have discovered the same species in `semine corrupto' (as a dried out husk?).'' (reproduced from \citep{Huygens}). \textbf{C}: 8 of the 10 volumes of the collected correspondances of Christian Huygens as prepared for the Dutch Society of Sciences and published from 1888-1905)}
\includegraphics[width=\textwidth]{Christiaan_Huygens_combined_figure.pdf}
\end{figure}



``The story is told that Sir Robert Peel, the Prime Minister, visited Faraday in the laboratory of the Royal Institution soon after the invention of the dynamo.
Pointing to this odd machine, he inquired of what use it was. Faraday is said to have replied `I know not, but I wager that one day your government will tax it.'''
from \citep{PearceWilliams1965} potentially apocraphyl (it took until 50 or so years after the 1880s for this prophecy to become true)


\begin{figure}
    \caption{\textbf{A}: Schematic of the current best estimate of the tree of life demonstrating the 2D and 3D hypotheses,
dashed lines indicate multiple potential branch location, arrowed lines demonstrate known endosymbiotic events (based on work reviewed in \citep{Gribaldo2010})
\textbf{B}: Schematic of the current known eukaryotic portion of the tree of life (based on work reviewed in \citep{Burki2014,Adl2013},
\textbf{C}: Schematic of phylogeny of the ciliates (based on work by \citep{Bachvaroff2011,},
\textbf{D}: Schematic of phylogeny of the green algae (based on work reviewed in \citep{Leliaert2012,}}
    <++>
\end{figure}

\begin{figure}
    \caption{A schematic of known endosymbioses across the tree of life based on work by \citep{Gribaldo2010,Wrede2012,vonDohlen2001}}
\end{figure}
%\subsection{History}
%
%\subsection{Biology}
%
%\subsection{History}
%
%
%
%\section{Background}


\section{\textit{Micractinium reisseri}}
From H\"ammerling's pioneering research with \textit{Acetabularia} in the 1930s 
which played a fundamental role in the process of 
of unravelling central dogma with the first clear demonstration of the 
developmental role the nucleus plays, anticipating the discovery of mRNA by 30 years
\footnote{CITATION}, research on green algae has made fundamental contributions
to basic biological understanding

\subsection{\textit{Paramecium bursaria – Chlorella}}

While the perialgal vacuoles of PbMr are not necessarily as specialised as the 
mucoidal bacteriole of host mealybug cells such as those described \citep{vonDohlen2001}

%\textit{Paramecium bursaria} or the `green \textit{Paramecium}' is an alveolate ciliate protist that forms a secondary photosynthetic endosymbiosis with several species of \textit{Chlorella}, a green algae.  
%Each \textit{Paramecium bursaria} cell contains \textasciitilde 300 endosymbiotic \textit{Chlorella} maintained in individual perialgal vacuoles in a stable, heritable (roughly equal partitioning of endosymbionts into daughter host cells), mutually beneficial, faculative endosymbiosis. 
%With the exception of the primary endosymbioses of the archaeplastida, almost all oxygenic phototrophs are believed to have arisen via secondary or higher order symbioses [Hoshina \& Imamura, 2009].  
%Therefore, understanding of the mechanisms and evolution of this relationship is key to our understanding of the evolution of several groups of the eukaryotes. The \textit{Paramecium-Chlorella} system is considered by many researchers to form a very good model for investigating the evolution and molecular basis of secondary symbiosis [Hoshina \& Imamura, 2009].  
%Historically, \textit{Paramecium} are one of the most studied group of protists and thus there is an extensive existent literature on the molecular biology of this group and the \textit{Paramecium bursaria – Chlorella} endosymbiosis in particular.  
%Biologically, the faculative nature of this relationship means the system putatively represents a nascent endosymbiosis and thus provides a means of studying endosymbiosis before metabolic co-dependence becomes fixed. 
%Another major experimental benefit of the faculative relationship is the host and endosymbiont can be separated, cultured independently and then the symbiosis re-established.  
%This allows experimental predictions to be readily tested via established molecular biological techniques such as RNAi.
%
%The \textit{Paramecium - Chlorella} endosymbiosis is established when \textit{Chlorella} is phagocytosed by the serially phagotrophic \textit{Paramecium} and is then able to escape the digestive vacuole.  
%For this escape to take place, the endosymbiont must initially resist acidification caused by acidosome fusion with digestion vacuole.  
%If the endosymbionts are able to resist this acidification they begin, through an unknown mechanism, to `bud-off' from the initial phagosome into a new vacuole.  
%This new perialgal vacuole (PV) is released into the cytoplasm and each PV contains an individual \textit{Chlorella} cell [Kodama \& Fujishima, 2009].
%The PV appears resistant to lysosome fusion and further digestive steps suggesting molecular modification of the vacuole membrane [Johnson, 2011]. 
%These perialgal vacuoles then bind the host cortex and compete for attachment with host structures known as trichocysts [Kodama, 2012] in a region with low to no lysosome activity [Kodama \& Fujishima, 2009].  
%This suggests the observed resistance to lysosome fusion may be a by-product of localisation. 
%As few as a single algal cell can infect the host [Weiss, Ayala 1976] however, the majority of \textit{Chlorella} are digested especially non-competent strains [Kodama, Nakahara, and Fujimura 2007].  
%Furthermore, it has been established that \textit{Chlorella} strains are fairly host-specific.  
%For example, Summerer et al in 2007 showed that \textit{Chlorella} isolated from other ciliates were able to establish endosymbioses with \textit{P. bursaria} however, those isolated from cnidarian \textit{Hydra} were not.  
%This paper also showed \textit{P. bursaria} favours its symbiotic partner over those isolated from other ciliates when given the choice, this suggests specific adaptations have taken place between host and endosymbiont [Summerer et al, 2007].
%Free-living \textit{Chlorella} strains do rarely establish endosymbioses with \textit{Paramecium} [Karakashian 1959], however they are generally only able to infect fewer \textit{Paramecium} and establish much smaller endosymbiotic populations within the host than the symbiont strains [Karakashian 1965].
%
%Once established, the symbiosis appears to be mutually beneficial with an observed flux of amino acids and CO$_{2}$ to the endosymbiont and oxygen and photosynthate (principally maltose) to the host as a function of light levels [Karkashian, 1963]. 
%The extent of this endosymbiosis is such that \textit{Chlorella} is capable of supporting \textit{Paramecium} in media without its typical bacterial food-stocks and conversely the \textit{Paramecium} is capable of supporting the phototrophic \textit{Chlorella} in the dark for \textasciitilde 2 weeks (or up to 51 endosymbiont cell divisions) suggesting considerable bi-directional nutrient flux [Siegel, 1960, Karkashian 1963]. 
%It should be noted that for longer periods in the dark or when a bacteria-free culture is used in the dark the host will digest the endosymbionts [Parker, 1927].   
%From an ecological perspective, this endosymbiosis can be considered as a means of acquired phototrophy (or mixotrophy), a tactic believed to be advantageous for survival in patchy oligotrophic environments by providing fixed carbon to cover respiration requirements [Putt, 1990].  
%This is largely supported by studies, such as Karkashian's 1963 paper, showing that with a sufficient concentration of bacterial feedstock in the media the growth rate of asymbiotic \textit{Paramecium} ('bleached') and \textit{Paramecium} with \textit{Chlorella} endosymbionts are largely equal.
%This threshold is estimated to lie between $10^{6}$ and $10^{7}$ bacteria per ml.  
%However, as this is generally a much greater concentration than found in the natural environments of \textit{P. bursaria} the endosymbiosis offers a considerable adaptive advantage to the host [Karkashian 1963].  
%As temporary acquisition of phototrophy is estimated by some research [Raven, 1997] to be less energetically costly than the permanent maintenance of plastids (via endosymbiosis or kleptoplasty) within the host this indicates that this endosymbiosis likely provides other host benefits beyond just the energetics of acquired phototrophy. These include: 
%\begin{itemize}
%  \item Exploitation of low oxygen environments by the host (as the photosynthesising endosymbiont is capable of providing oxygen to the host [Reisser, 1980])
%  \item Photoprotection and protection against 257nm and 282nm UV radiation potentially via endosymbiont pigmentation and localisation to shield host nuclei [Sommaruga \& Sonntag, 2009, Summerer, 2009, Miwa, 2009].  This is especially important as the AT-rich \textit{Paramecium} genome is likely prone to UV-damage via the formation of cyclobutane thymine dimers [Sommaruga \& Sonntag, 2009].
%  \item Protection against predation [Berger, 1980]. The exact mechanism by which this occurs is unknown, however, it has been observed that mixotrophic ciliates are able to move in rapid `jumping' movements. This is hypothesised as being an energetically costly escape reaction made possible by sugar-rich photosynthate mixotrophic ciliates gain from their algal endosymbionts [Perez et al. 1997]. Intriguingly, this protection against predation occurs despite endosymbiont displacement of trichocysts (defensive cellular structures) for attachment to the ciliate cortex [Kodama, 2011].
%  \item Protection against chemical toxins, for example symbiotic \textit{Paramecium} have a much higher survival rate (96\%) to 0.5 mM nickel chloride (NiCl$_{2}$) than asymbiotic \textit{Paramecium} via an undetermined mechanism [Miwa, 2009].
%  \item Increased thermotolerance (tested at $42^{o}$C) [Miwa, 2009], again, by unknown mechanisms but potentially related to the undefined means of perialgal vacuole attachment to the cell cortex.
%  \item Protection against excessive oxidative burden (potentially due to endosymbiont dismutases and catalases) [Hortnagl \& Sommaruga, 2007] and hydrogen peroxide (hypothesised by Miwa as being due to the improved energetics of the symbiotic host) [Miwa, 2009]
%\end{itemize}
%
%In return, the endosymbiont also appears to gain several advantages including a generally much increased level of photosynthetic activity [Sommaruga \& Sonntag, 2009]:
%\begin{itemize}
%  \item CO$_{2}$ from the host [Parker 1927].
%  \item Nitrogen supply [Johnson, 2011].
%  \item Amino acids including L-glutamine (likely an important nitrogen source) [Reisser \& Widowski, 1992] and L-arginine, L-asparagine, L-serine, L-alanine and glycine [Kato \& Imamura, 2009].
%  \item Host supplied divalent cations such as K$^{+}$, Mg$^{2+}$, and Ca$^{2+}$. All of which have key roles in photosynthesis [Kato \& Imamura, 2009].
%  \item Protection against \textit{Paramecium bursaria – Chlorella} Virus (PBCV) [Yaschenko, 2011] a large isocahedral dsDNA, 330kbp virus with 133-genes that lyses symbiotic Chlorella when isolated from the host [van Etten, 1982].  This potentially occurs by preventing contact between PBCV and the endosymbiont.
%  \item Effective photo-accumulation and increased mobility [Neiss, 1982]  
%\end{itemize}
%
%This exchange of materials between host and endosymbiont is regulated by an effective biochemical 'bartering' system with numerous feedback cycles (see summary from book).
%For example, the release of endosymbiont photosynthate is dependent on Ca$^{2+}$.  
%This ion is provided by the host and also has a role in the up-regulation of photosynthesis (as proxied by oxygen evolution) [Kato \& Imamura, 2009].     
%Once photosynthate is released into the PV lumen endosymbiont H$^{+}$-ATPases are activated which allow the generation of the H$^{+}$ gradient necessary for endosymbiont uptake of host-provided amino acids via a set of amino acid-proton symporters [Camoni et al. 2006].  
%This proton gradient will potentially lead to further photosynthate release due to observed pH-dependence of this [Kato \& Imamura, 2009]. 
%As we can see the more photosynthate supplied to the PV lumen the greater the uptake of provided nitrogen sources.  
%Intriguingly, from experiments using cycloheximide to selectively interrupt endosymbiont but not host protein synthesis it appears that the maltose transporter that is responsible for export of photosynthate from the PV lumen into the host cytoplasm is endosymbiont derived [Muscataine, 1967]. 
%However, unless photosynthesis is also inhibited (using DCMU) the build up of photosynthate without exportation in the PV triggers the swelling of the vacuole up to 25x its original size.  
%This removes the vacuole from the region in which it is protected from lysosome fusion and leads to the digestion of the endosymbiont [Kodama \& Fujishima, 2009]. 
%So, here we can see further regulation of the relationship – in which the endosymbiont is degraded if it does not release photosynthate to the host.
%
%On top of this system of secretion, uptake and feedback (summarised in REF FROM BOOK), there have also been several other observed regulatory interactions between host and endosymbiont.  
%The most apparent of these are the synchronising of cell division and circadian rhythms between host and endosymbiont [Miwa, 1996] with endosymbiotic \textit{Chlorella} sufficient to recover a circadian rhythm in arrhythmic \textit{Paramecium} mutants [Miwa, 2009].  
%This regulation of the timing of cell division for both members of the system appears well co-ordinated and takes place in such a way that neither host or endosymbionts outgrow one another [Kadono et al. 2004; Takahashi et al. 2007].
%
%% Mieosis turned off %
%
%
%\subsection{RNA-Seq}
%Transcriptomics and specifically RNA-Seq is one of the key investigative techniques that will be used in this project.  
%By extracting and sequencing mRNA expressed by host and endosymbiont during day and night conditions we will identify key genes involved in endosymbiotic interactions.
%RNA-seq is a second-generation sequencing technology developed to perform high-throughput short-read sequencing of cDNA generated from extracted RNA [Grabherr, 2011]. 
%As ribosomal RNA forms such a significant portion of cellular RNA it is necessary to remove it before sequencing in order to make meaningful inferences of transcript expression.  
%This can be achieved either by ribosomal depletion in which ribosomal sequences are bound by adaptors which in turn bind to magnetic beads allowing the mechanical removal of these molecules.  
%This methodology allows the preservation of smaller non-coding RNAs present in the system. However, as we are principally interested in transcripts and are operating in a eukaryotic system with poly-adenylated mRNA we can efficiently isolate mRNA via a poly-A selection, in which poly-T primers are used in reverse-transcription of mRNA into cDNA.
%From this point RNA-Seq proceeds in the same manner as other next-generation sequencing methods. 
%We are using a method known as paired-end sequencing (available on the Illumina HiSeq 2000 platform) in which 76bp reads are sequenced at either end of a 200-500bp insert. 
%This improves the ease of transcriptome assembly by providing additional positional information when producing de-Bruijn graphs. 
%
%\subsection{Transcriptome Assembly}
%Assembly is the process by which the short high-throughput sequencing reads are combined to recover the longer transcripts from which they have been randomly sampled.  
%In the simplest sense, this process relies on the quality score weighted alignment of reads.  
%Transcriptome assembly can be done in two ways: referenced (in which reads are aligned to a reference genome), and de-novo (in which the reads are assembled without a reference).  
%De-novo assembly are mainly conducted using k-mer based de-Bruijn graph creation.  
%Very briefly, reads are subdivided into k-mers (\textasciitilde $15-30$ nucleotide sequences) to aid in computation and then assembled in graphs based on k-1 overlaps between k-mers.
%With the exception of repeated regions the transcript is recovered by finding the shortest path around the graph visiting all vertexes once (i.e. a hamiltonian path). 
%In reality, assembly algorithms weight the construction of these k-mers and graphs using sequence quality scores and a large number of established heuristics.
%Referenced assemblies are often considered sub-optimal in some eukaryotic organisms due to alternative splicing and other transcript modifications not directly represented in the genomic sequences.  
%Despite this, for a \textit{Paramecium - Chlorella} transcriptome, host transcripts could still potentially be assembled as the closest available genome sequence is that of the \textit{Paramecium tetraurelia} macronuclei in which most intronic elements have been spliced out [Mayer \& Forney 1999].  
%However, using this genome as a reference for the assembly of host transcripts may be highly misleading due to the fact that this organism does not harbour \textit{Chlorella} endosymbionts and thus may lack the genes required to maintain this system and also has undergone two rounds of whole genome duplication since sharing a common ancestor with \textit{Paramecium bursaria} potentially leading to high levels of sequence divergence.  
%On the other hand, endosymbiont transcripts could be assembled using the endosymbiotic \textit{Chlorella} NC64A genome [Blanc et al., 2010] once the relationship between this organism and the endosymbiont within the CCAP 1660/12 culture is determined phylogenetically and is determined as meaningfully close.  However, the potential issue of alternative splicing would occur if we were to use this genome sequence.  
%Therefore, our initial assembly was conducted de-novo (i.e. without any reference genomes) in order to attempt to avoid biasing due to these reference genomes.   
%
%
%\subsection{Phylogenetics}
%The next key method in this project is that of phylogenetics as it is an effective tool with which to investigate the evolutionary ancestry of the genes recovered in the transcriptome and to identify the likely origin (host, endosymbiont, contaminant) of these transcripts.  
%It can also be used to identify potential horizontal gene transfer events between host and endosymbiont by searching for single gene/transcript phylogenies that have an incongruent branching pattern compared to established species trees.    
%The key stages in a phylogenetic analysis are that of alignment (in which homologous sites in the sampled sequences are aligned with one another), masking (in which sites which are evolutionarily informative – can be determined to be homologous but also non-invariant are selected), model selection (in which a sequence evolution model is selected using criteria such as Akaike's information criterion which penalise increased parameterisation but attempt to maximise model likelihood), and finally, phylogenetic reconstruction (in which the selected evolutionary models are applied to the dataset using principally maximum-likelihood or Bayesian methodologies in order to reconstruct the likely evolutionary relationships in the masked sequences). 
%
%\subsection{Annotation}
%In order to identify proteins putatively involved in endosymbiosis – namely transporters and secreted proteins and to identify proteins likely involved in their regulation it is necessary to functionally annotate assembled transcripts.  
%This was achieved in a variety of ways including Hidden-Markov Models or HMMs (via HMMer, PFAM and other similar resources), alignment (via BLAST), KEGG Ontology and Gene Ontology annotation (via tools such as the KEGG Automated Annotation Server or BLAST2Go). 
%HMMs are a commonly used tool in sequence identification and depending on their training are generally more sensitive than alignment based methods [Eddy., 2011].
%This will be further discussed within the methods.
%
%\section{Methods}
%\subsection{Culturing}
%
%\subsubsection{Media}
%\textit{Paramecium bursaria – Chlorella} CCAP1660/12 cultures were obtained from the UK Culture Collection of Algae and Protozoa and were maintained in New Cereal Leaf-Prescott Liquid (NCL) media (4.3g/l CaCl$_{2}$.2H$_{2}$0, 1.6g/l KCl, 5.1g/l K$_{2}$HPO$_{4}$, 2.8g/l MgSO$_{4}$.7H$_{2}$O, 1g/l wheat bran, gravity filtered via GF/C paper and autoclaved) and stored in a lit incubator at 15$^{o}$C. [\url{http://www.ccap.ac.uk/media/documents/NCL.pdf}].  
%Cultures were sub-cultured every 2 weeks with fresh NCL media and were inspected using light microscopy to assess culture health.  
%
%\subsubsection{Harvesting cells for transcriptome analysis}
%To harvest \textit{P. bursaria} and minimise the number of bacterial prey species from the culture, \textasciitilde $10^{6}$ cell aliquots were strained through 40$\mu m$ sieves, filtered on 10$\mu m$ nylon filters, before finally being filtered on 8$\mu m$ TETP polycarbonate filters using a low-pressure filtration pump.  
%Collected samples were either immediately quick-frozen in liquid nitrogen for storage ($-20^{o}$C for short-term storage and $-80^{o}$C for longer storage) or harvested by centrifugation.  
%In order to investigate the two main metabolic states of the symbiosis (i.e. under light conditions during active photosynthesis and in the dark when no photosynthesis is taking place) samples were extracted 5 hours into the light and dark phase of the day/night cycle.
%
%\subsubsection{Culture Health}
%To ensure extracted RNA was representative of healthy and interacting host and endosymbionts care was taken to minimise the number of dead/dying cells from which RNA was extracted.  
%In order to do this, a subsample was taken from each culture during the process of harvesting and scored for dead/dying cells.  
%Cell assays were formed by taking 1-2ml of each harvest cell pellet and fixed using 40$\mu l$ Lugol's solution (0.5g I$_{2}$ and 1g KCl in 8.5ml of MilliQ water). D
%ead/dying cells were identified as broken or puckered cells and counted using light microscopy.  
%Samples containing >10\% dead/dying cells were discarded and no RNA extracted from them.
%
%\subsubsection{RNA Extraction and Isolation}
%In order to lyse collected samples, cells were washed from the filter or the pellet was resuspended in 1ml TriReagent (Sigma) heated to $60^{o}$C. Cells were vortexed with sterile 300$\mu m$ glass-beads for 15s, incubated at room temperature for 10 minute, vortexed for 15s, quick-frozen in liquid nitrogen and stored at $-20^{o}$C before further processing.  
%Samples were defrosted, vortexed for 15s, placed in a heat-block set to $60^{o}$C for 10 minutes while continuing to be vortexed, removed from heating and vortexed again for 15s.  
%RNA was extracted by adding 0.2ml of Chloroform to the glass-bead-trizol-sample solution, shaking for 15s, incubating for 5 minutes at room temperature and centrifuging at 12,000g for 15 minutes at $4^{o}$C.  
%The upper-phase was then transferred to an RNAse-free 1.5ml tube and an equal volume (\textasciitilde$0.5$ml) of isopropanol was added before shaking for 15s.  
%The isolated RNA was then incubated at $-20^{o}$C for 10 minutes (up to several hours) before being collected as a pellet using a centrifuge at 10,000g for 10 minutes at $4^{o}$C (supernatant was discarded). 
%The RNA pellet was then washed with 1ml of 75\% ethanol and centrifuged twice at 10,000g for 10 minutes at $4^{o}$C with the supernatant being discarded after each centrifugation.  
%Pellet was dried before being resuspended in 100$\mu l$ of RNAse-free water.  
%The RNA was then cleaned further using the Qiagen RNeasy clean-up kit before being assessed for quality using ND-1000 (NanoDrop) and BioAnalyzer (Agilent).
%
%\subsection{RNA-Seq}
%\subsubsection{cDNA generation and library preparation}
%
%Purified RNA from day and night samples were pooled into a single day and a single night sample due to limitations on the quantity of extracted RNA. 
%These two samples were then prepared for sequencing using the TruSeq Stranded RNA preparation kit (Illumina).  
%mRNA was selected using a poly-A selection method before being fragmented, cleaned-up using ethanol and then reverse transcribed into single-stranded cDNA using random hexamer primers. 
%The prepared ds cDNA was then end-repaired or cleaved to produce blunt-ended fragments.  
%By the addition of a trailing A-base to these fragments sequencing adaptors were added using an overlapping complementary T-base.  
%Fragments were then denatured and amplified before cluster generation using the standard Illumina sequence library preparation methods [\url{http://res.illumina.com/documents/products/datasheets/datasheet_truseq_sample_prep_kits.pdf}].
%
%\subsubsection{Sequencing}
%
%Clonal cluster generation by bridge amplification was conducted on the cBot (Illumina) platform using the TruSeq Paired-End cluster kit (Illumina).  
%The samples were run in individual lanes on a HiSeq 2000 8-lane flow-cell (Illumina) to produce 76bp paired-end reads. 
%Sequencing was conducted at the University of Exeter sequencing service.
%
%\subsubsection{Assembly}
%The paired-end fastq reads from the sequencing experiment were trimmed to sequences of sufficient quality and sequencing adaptors removed. 
%Reads from day-and-night runs were assembled separately as well as a combined assembly of all the reads. 
%As the \textit{Paramecium bursaria} 1660 genome or the \textit{Chlorella} endosymbiont (CCAP12) have not yet been sequenced de-novo transcriptome assembly was conducted (further explanation for this in background section).  
%This was done using both the Oases package [Schluz, 2012] from European Bioinformatics Institute [\url{http://www.ebi.ac.uk/~zerbino/oases/}] and the Trinity package [Haas, 2013] provided by the Broad Institute [\url{http://trinityrnaseq.sourceforge.net/}] using a high-memory computing cluster at the University of Exeter.  
%Assemblies were compared and a single assembly methodology was selected for further downstream analysis. 
%
%\subsubsection{Saturation}
%In order to assess whether increased sequencing depth would recover further transcripts a read-saturation analysis was conducted.  
%This was achieved by developing a program to repeatedly take random subsamples of the paired-end reads (from the day sample, the night sample and the combined pool) and assessing the number of the assembled transcripts that could be successfully recovered using these reads.  
%This was done with the aid of a short-read aligner bowtie-2 [Langmead, 2012], standard GNU/Linux command line tools, and a custom python script.  
%The results were plotted using python+matplotlib and assessed for saturation in transcript recovery
%
%\subsubsection{Extracting Likely Coding Regions}
%As host and endosymbiont use different codon tables it was necessary to identify ORFs in the assembled transcripts using both tetrahymena (i.e. Host/Paramecium) and universal (endosymbiont/Chlorella and bacterial contaminant/foodstock) encodings.  
%The ciliate `tetrahymena' encoding differs from universal codon look up tables in that UAA and UAG which are normally stop codons instead code for glutamine, this means \textit{Paramecium} only has a single stop codon (UGA) [Harper \& Jahn, 1989].
%This was conducted using the Trinity transcripts\_to\_best\_scoring\_ORFs tool [\url{http://trinityrnaseq.sourceforge.net/analysis/extract_proteins_from_trinity_transcripts.html}].  
%Initially, this tool identifies the longest ORF within each assembled transcript.  
%These longest ORFs are used to train a hexamer-based HMM which is then used to identify the ORFs with highest likelihood.  
%These ORFs are then translated into peptide sequences. 
%
%\subsubsection{Transcript Identification}
%Despite care being taken in filtering and washing-off bacteria from the \textit{Paramecium} prey food-stock and usage of poly-A selection, bacterial contamination was still present in our assembled contigs.  
%Therefore, it was necessary to develop a pipeline to identify the transcript origins to filter those transcripts originating from host and endosymbiont. 
%Initially transcripts were binned into their predicted source - Host (H), Unknown but likely Host (U(H)), Endosymbiont (E), Unknown but likely Endosymbiont (U(E)), Food (F), Unknown but likely Food (U(F)), and Unknown (U).  
%Each of the assembled transcripts were BLASTed against a database consisting of the following genomes: \textit{Chlorella} NC64A, \textit{Chlamydomonas reinhardtii}, \textit{Coccomyxa} C169, \textit{Paramecium tetraurelia}, \textit{Tetrahymena thermophila},  \textit{Arabidopsis thaliana}, \textit{Homo sapiens} (helping to identify contamination), \textit{Saccharomyces cerevisiae}, \textit{Schizosaccharomyces pombe}, \textit{Bacillus cereus} ATCC 14579, \textit{Escherichia coli} 536, \textit{Escherichia coli} O157 H-7, \textit{Salmonella typhimurium} LT2 and \textit{Escherichia coli} K-12 (the last 5 genome datasets helping to identify food bacterial genes).  
%The bins were classified as follows:
%\begin{itemize}
%  \item Endosymbiont (E): Transcript's highest scoring BLAST hit at -50E was to \textit{Coccomyxa}, \textit{Chlamydomonas} or \textit{Chlorella}.  Or transcript's highest scoring hit at -20E was one of those species and the longest likely coding region in the transcript was using the universal codon table. 
%  \item Unknown but likely Endosymbiont (U(E)): -10E hit to \textit{Coccomyxa}, \textit{Chlamydomonas} or \textit{Chlorella} and longest likely coding region was using the universal codon table. Or transcript's highest hits at -10E and -20E were to one of those genomes regardless of coding region presence.
%  \item Host (H): Transcript's highest hits at -50E were to \textit{Paramecium tetraurelia} or \textit{Tetrahymena thermophila}.  Or highest hit at -20E was one of those species and longest likely coding region was using the Tetrahymena codon table.
%  \item Unknown but likely Host (U(H)): -10E hit to \textit{Paramecium tetraurelia} or \textit{Tetrahymena thermophila} and longest likely coding region was was using the Tetrahymena codon table. Or transcript's highest hits at -10E and -20E were to one of those genomes regardless of coding region presence.
%  \item Food (F): Transcript's highest scoring BLAST hit at -50E was to one of the \textit{E. coli} species or \textit{Salmonella}.  Or transcript's highest scoring hit at -20E was one of those species and the longest likely coding region in the transcript was using the universal codon table.
%  \item Unknown but likely Food (U(F)): -10E hit to one of the \textit{E. coli} species or \textit{Salmonella} and longest likely coding region was using the universal codon table. Or transcript's highest hits at -10E and -20E were to one of those genomes regardless of coding region presence.
%  \item Unknown (U): highest scoring hits to \textit{Arabidopsis}, \textit{Homo sapiens}, \textit{Saccharomyces} or \textit{Schizosaccharomyces} or any sequence not fitting into the above categories.
%\end{itemize}
%
%In order to assess the accuracy of the `binning' of endosymbiont sequences and to estimate how many host or endosymbiont transcripts we would accidentally discard using these categories we used a phylogenetic analysis pipeline to verify the likely ancestry and therefore point of origin within the endosymbiosis of each sequence.  
%A script was created used the translated transcript peptide sequences to search a database of over 900 representative genomes using BLASTp.  This script recovered all sequences above an E-value of $10^{-5}$. 
%These sequences were then automatically aligned and masked using MUSCLE [Edgar, 2004] and TrimAL [Capella-Gutierrez, 2009]. 
%Fast maximum-likelihood trees were then generated for each masked alignment using FASTREE2 program [Price, 2010].  
%These trees were then converted to scalable vector graphics and using the NCBI taxonomy API each branch was colour coded depending on taxonomic group.  
%This allowed for quick manual inspection to determine the likely origin identity of the initial transcript.  
%For example, if a transcript sequence was found to branch largely within archaeplastida species it was considered as likely originating from the endosymbiont.  
%Likewise, if it branched largely among ciliate species it was considered as likely host-derived.  
%Continuing in the same vein, those sequences considered as being derived from bacterial food-stocks were considered as those that branched largely within bacterial clades and those sequences that branched with no clear pattern were considered to be `Unknown'.  
%Any transcripts where endosymbiont sequences were found to branch incongruently with accepted topologies (especially within the ciliate or bacteria) were determined to be showing signs of potential horizontal gene transfers and were recorded for further analysis.  
%For example, if a transcript branched with another endosymbiotic algae (particularly \textit{Chlorella} NC64A) within a clade of ciliates this would be considered a potential HGT event to be further investigated.
%The entire initial `Endosymbiont Bin' (2,825 transcripts), the `Unknown but likely Endosymbiont Bin' (600) and the `Unknown Bin' (2,096) were phylogenetically tested using this pipeline.  
%However, owing to the size of the initial `Food Bin' (5,129) and `Host Bin' (27,907) it wasn't possible to exhaustively test all transcripts in these two bins therefore 2000 transcripts were randomly selected from each them and tested using the pipeline. 
%In the cases where too few sequences were recovered to produce a phylogeny (<3 including seed) the source proteomes of the sequences that were recovered were assessed. For example, if the recovered sequences were from the archaeplastida that was considered sufficient evidence to class that transcript as belonging to that `Endosymbiont Bin' and so on.  If no other sequences were recovered using BLASTp then the transcript was placed into the `Unknown Bin'.
%The change in initial binning after phylogenetic verification were then plotted using python+matplotlib and an estimate for the numbers of unidentified endosymbiont transcripts was made.
%
%\subsubsection{Identification of putative secreted and putative transporter proteins}
%Initial transcript functional annotation was conducted via BLAST and mapping Gene Ontology terms using the Blast2GO application [Conesa, A. et al., 2005] [\url{http://www.blast2go.com/b2ghome}].  
%In order to more accurately identify the likely proteins involved in the maintenance of the \textit{Paramecium-Chlorella} photosynthetic endosymbiosis we attempted to identify all transporter and secreted proteins within the verified endosymbiont and host transcript complements. A permissive set and conservative subset of host and endosymbiont transporter and secreted proteins were identified using a combination of pipelines.
% Putative transporter proteins were identified as all coding sequences within each bin containing 1 or more transmembrane domains (as identified by the HMM implemented in the program tmhmm 2.0c  [Krogh et al., 2001, Sonnhammer et al., 1998]).  
%The conservative subset of transporter proteins were those identified as containing 4 or more TM domains.  
%The predicted host and endosymbiont secretome was determined by combining the outputs of several established and widely-published programs for identification of secreted proteins - SignalP 4.1 [Petersen et al., 2011], tmhmm 2.0c, TargetP 1.1 [Emanuelsson et al., 2000], and WoLFPSORT 0.2 [Horton et al., 2007] (with the assistance of some accessory scripts from from UCSC [\url{http://hgdownload.cse.ucsc.edu/admin/exe/}] and the FASTX-toolkit [\url{http://hannonlab.cshl.edu/fastx_toolkit/index.html}]). 
%Input sequences were checked for the presence of N-terminal signal peptides primarily via the detection of signal peptide cleavage sites by the HMM implemented in SignalP 4.1. 
%Mature peptides (i.e. those with the detected signal peptide cleaved) output from SignalP 4.1 were then processed in tmhmm 2.0c to detect and remove any sequences containing transmembrane domains. 
%The output from these two stages were then combined with all initial input sequences detected as secreted (S) by TargetP 1.1 and as having an extracellular localisation via WoLFPSORT 0.2 to produce a single fasta formatted output file containing the entire predicted secretome, i.e. a set of proteins predicted to be secreted by all three bioinformatic methods. 
%This therefore identified the conservative subset of host and endosymbiont secreted proteins. 
%A second list of putative secreted proteins was identified as any protein predicted to be secreted by any one of either SignalP 4.1, WoLFPSORT 0.2, or TargetP 1.1. It must be noted that such prediction approaches are only as good as the quality of each individual transcript assembly. 
%Genes with low transcript levels are likely to have low coverage and assembly quality potentially leading to false negative detection of N-terminal signal peptides. 
%This was investigated in more detail below.
%
%\subsubsection{Coverage}
%As the identification of the secreted proteins hinged upon the recovery and accurate prediction of N-terminal signal peptides it was necessary to assess the position-specific read coverage across the transcript specifically at the C- and N- termini. 
%This was achieved by developing a script to count and average (median) the number of reads aligning to each nucleotide in both tetrahymena and universal (i.e. host and endosymbiont) likely coding regions using a custom python script.  
%This was done for all tetrahymena and universal coding regions in transcripts, all phylogenetically confirmed host and endosymbiont binned sequences, conservatively and permissively identified putative secreted and transporter proteins in the `Host Bin' and `Endosymbiont Bin'. 
%Results were then plotted using python+matplotlib.
%
%\subsubsection{Endosymbiont Species Confirmation}
%In order to confirm the identity of the endosymbiont species from CCAP two analyses were conducted. 
%Firstly, the endosymbiont binned transcripts were searched for any 18S rRNA sequences present in the transcriptome.  
%The two identified fragments of 18S sequence were aligned against a database of green algal 18S sequences (from [Hoshina and Imamura, 2008]) using MUSCLE.    
%The multiple-sequence alignments were then masked manually using Seaview [Gouy, 2010].  
%Evolutionary model was estimated using Prottest3 [Dariba, 2011] for each masked alignment and as highest likelihood models were identical (LG+$\Gamma$+I) alignments were concatenated into a single masked alignment.  
%Bayesian and Maximum Likelihood phylogenetic reconstructions were then conducted using an un-partitioned suggested model and both MrBayes v3 and PhyML [Guindon, 2010].  
%The ML phylogeny was plotted as an SVG and key taxa were highlighted.
%
%Secondly, \textit{Chlorella} specific primers were used to amplify, clone and sequence the identifying ITS2 region.  
%This was achieved by using the Qiagen DNeasy Plant DNA extraction kit after bead beating to disrupt the \textit{Chlorella} cells from a culture aliquot. 
%Using \textit{Chlorella} specific primers (from [Hoshina and Imamura, 2008]) PCR was used to amplify the \textit{Chlorella} ITS2 region from the  extracted DNA samples.  
%These extracted samples were then sub-cloned using the StrataClone-TA-cloning kit (StrataClone) and blue-white screened for successful insert ligation. 
%9 white colonies were selected and prepared for sequencing using the SV Wizard MiniPrep kit and PCR ligation of M13 universal sequencing primers.  
%Sequences were then Sanger sequenced via Beckman-Coulter Sequencing Service.  
%The returned sequences were then used to reconstruct a phylogeny of ITS2 region using BLAST against species outlined by [Hoshina and Imamura, 2008], aligned using MUSCLE, masked in Seaview, model testing in Prottest3 and reconstructed using MrBayes and PhyML.
%
%\section{Results}
%\subsection{RNA-Seq Assembly}
%The comparison of De-novo transcriptome assemblies can be seen in assembly summary below.
%The metrics clearly show the Trinity assembly is far superior to that of the Oases assembly with over a 50\% reduction in total contig number while recovering consistently longer contigs (a histogram of the Trinity contigs lengths and distribution can be see in contig lengths). 
%All further analyses were done using Trinity based pooled assembly.
%
%\begin{table}
%\centering
%\begin{tabular}{| c || c | c |}
%\hline
%\textbf{Assembly Metric} & \textbf{Oases Assembly} & \textbf{Trinity Assembly} \\
%\hline
%\textbf{Min Contig Length:} & 100 & 201\\
%\textbf{Max Contig Length:} & 16,202 & 17,729\\
%\textbf{Mean Contig Length:} & 648.90 & 959.32\\
%\textbf{Standard Deviation of Contig Length:} & 939.04 & 1080\\
%\textbf{N50 Contig Length:} & 1,368 & 1,621 \\
%\textbf{Number of Contigs:} & 117,570 & 48,003\\
%\textbf{Number of Contigs >=1kb:} & 22,225 & 14,774\\
%\textbf{Number of Contigs in N50:} & 14,977 & 8,060\\
%\textbf{Number of Bases in All Contigs:} & 76,290,606 & 46,050,097\\
%\textbf{Number of Bases in All Contigs >=1kb:} & 46,695,005 & 31,602,626\\
%\textbf{GC Content of Contigs:} & 28.99\% & 30.97\% \\
%\hline
%\end{tabular}
%
%\caption{Pooled read RNA-Seq De-Novo Assembly Comparison}
%\label{tab:assembly_summary}
%\end{table}
%
%The saturation analysis (in which we test whether further sequencing depth would recover more transcripts) is shown in saturation.  
%As can be seen the combined reads quickly trend to an asymptote suggesting the majority of transcripts were recovered at this sequencing depth.
%
%\subsection{Transcript Analysis}
%The initial identification and binning of recovered transcripts into host and endosymbiont categories was tested using a phylogenetic approach. 
%The results of this analysis were plotted in phylogenetic confirmation.  
%This demonstrates that the initial bin identifications were accurate for endosymbiont (\textasciitilde92\%) and food ( \textasciitilde94\%) derived transcripts.  
%Of the bins that were too large to be comprehensively phylogenetically tested (`Host Bin' and `Food Bin') we screened a subset of 2000 transcripts using a phylogenomic approach. 
%There were almost no detected endosymbiont sequences in 2000 randomly selected `Food Bin' transcripts therefore the `Host Bin' was the only cause for concern in containing misidentified endosymbiont transcripts.  
%\textasciitilde 2\% of the 2000 randomly sampled `Host Bin' transcripts were found to be wrongly identified endosymbiont sequences, assuming this random sample is representative of the `Host Bin' in general this suggests approximately 500 endosymbiont transcripts are misidentified as host sequences (within the 27,907 host sequences).  
%A key result here is that the bins that are to be discarded as contaminants (i.e. `Food Bin' and `Unknown Bin') have been either exhaustively checked and all host or endosymbiont transcripts they contain recovered or, as in the case of the `Food Bin', there are minimal sequences to be found.  
%This, combined with the saturation analysis allows us to be confident that we have recovered the maximum possible (even if \textasciitilde 500 endosymbiont are misidentified as Host) host and endosymbiont expressed transcripts during day and night conditions.
%
%In order to assess the accuracy of secretome predictions it was necessary to assess terminal coverage of likely coding regions.  
%The median coverage is plotted in coverage.  The key concern that this plot raises is that of poor terminal coverage throughout the `Host Bin' and `Endosymbiont Bin'. 
%As expected the identified secreted proteins (as these predictions largely hinge on detection of terminal signalling peptides) demonstrate a greater coverage suggesting on average only those transcripts with high terminal coverage were identifiable as secreted.  
%Due to the general low level of terminal coverage in the bins as a whole this raises the distinct possibility that many secreted proteins are undetected in the `Host Bin' and `Endosymbiont Bin'. 
%Transporter proteins are principally identified by the presence of trans-membrane domains so terminal coverage is not a concern for their identification. 
%For this reason, we are initially focusing on transporter proteins, however, there is a possibility to investigate secreted proteins using alternative strategies, in particular, proteomics.  
%Mass-spectrometry based proteomics may allow the identification (via peptide fingerprinting) of secreted proteins.
%
%\subsection{Preliminary Metabolic Mapping}
%Using the KEGG Automated Annotated Server KEGG Ontology annotations were recovered for the phylogenetically verified host and endosymbiont sequences and plotted onto the KEGG complete metabolic pathway (see metabolic mapping).  This is a rough preliminary tool that we can use to target later regulatory and interaction investigations for our candidate proteins.
%
%\subsection{Endosymbiont Species Confirmation}
%
%The phylogeny generated using detected Chlorella 18S sequences (see phylogeny) within the transcriptome clearly shows the concatenated sequence branching with high basal support (99.1\%) among the clade containing \textit{Chlorella} NC64A and endosymbionts 10 and 12. 
%The concatenated sequences branch directly (with low support) with \textit{Chlorella} 12 endosymbiont as expected from the CCAP provided \textit{Paramecium bursaria – Chlorella} CCAP 1660/12 culture.  
%ITS2 sequencing is still to be completed and thus this phylogeny cannot be reconstructed at this time, this will be completed within 2 months.
%Regardless, preliminary results support the identity of the endosymbiont as being the \textit{Paramecium bursaria} CCAP 1660/12 endosymbiont and the immediate sister to \textit{Chlorella} NC64A.  This piece of data is particularly important as \textit{Chlorella} NC64A has a complete published genome sequence [Blanc et al., 2010]. As NC64A strain is both endosymbiotic and a close-relative it may form a useful resource in annotating and further exploring the assembled transcriptomic data.  
%Potentially, this genome sequence may even be used for a referenced transcriptome assembly and allow improvement in assembled transcript quality.
%
%
%\section{Progress so far}
%\begin{itemize}
%  \item Verified the taxonomic identity of \textit{Chlorella} endosymbiont within \textit{Paramecium bursaria} sample provided by CCAP using molecular phylogenetics.
%  \item Cultured and purified transcripts from \textit{Paramecium-Chlorella} system in the two key metabolic states in their relationship – active and inactive photosynthesis during lit and dark conditions
%  \item Conducted a de-novo transcriptome assembly of the day/night transcripts
%  \item Assessed assembly quality and, in particular, average terminal coverage to determine the relative quality of annotation methods.
%  \item Phylogenomically identified host and endosymbiont transcripts, as well as filtering out those derived from contaminants such as bacterial food-stocks.
%  \item Annotated extracted transcripts with a focus on transporters and secreted proteins
%  \item Generated a guide-line differential expression analysis to help target further work
%  \item Assembled preliminary metabolic network of host and endosymbiont transcripts
%  \item Identified a set of transporters putatively involved in maintenance of endosymbiosis
%  \item Begun design of antibodies to determine the localisation of these transporters
%\end{itemize}
%
%
%\section{Further Work}
%The work conducted so far forms the ground-work necessary in order make meaningful insights into possible proteins involved in the maintenance of the endosymbiosis.
%Namely, confirmation of the endosymbiont species, day and night transcriptome sequencing and assembly, identification and verification of host and endosymbiont transcripts from the transcriptome and identification of transporter and secreted proteins within these subsets.  
%Using these greatly reduced numbers of sequences it is possible to manually identify candidates to further investigate, to ideally identify their role in maintaining the endosymbiosis.  
%Once we have developed our shortlist of candidates further we hope to verify our predictions using a variety of molecular methodologies:
%\begin{itemize}
%  \item Localisation - Polyclonal antibodies to confirm candidate is able to have a direct role in endosymbiosis (localised to perialgal vacuole or endosymbiont membrane for example) within the next 6 months.
%  \item Proteomics - Shotgun Mass Spectrometry - confirm presence of protein in system rather than just transcripts within 9 months.
%  \item Differential Expression Analysis - qPCR - to verify the regulation of candidate proteins at various points in the endosymbiosis (e.g. establishment, day, night) within the next 3-6 months.
%  \item RNAi - direct testing of predictions by determining the effect of candidate protein knock-out on state of endosymbiosis within 9 months.
%\end{itemize}
%
%\section{References}
%
%\noindent Achilles-Day, Undine E M,  Day, John G Isolation of clonal cultures of endosymbiotic green algae from their ciliate hosts, 2012 Journal of microbiological methods,  92,  3, 355, 7
%
% 
%\noindent Beijirnick, M. W. 1890 Kulturversuche mit Zoolchlorellen, Lichengonidien, und anderen Algen. Botan. Zeit., Jahrg. 48].   MISSING
%
% 
%\noindent Berger J  Feeding behavior of Didinium nasutum on Paramecium bursaria with normal or apochlorotic zoochlorellae 1980 . J Gen Microbiol 118 : 397 – 404
%
% 
%\noindent Chen, Tze-Tuan, Polyploidy in Paramecium bursaria, 1940,Genentics,  26(1)
%
% 
%\noindent Blanc G, Duncan G, Agarkova I, Borodovsky M, Gurnon J, Kuo A, Lindquist E, Lucas S, Pangilinan J, Polle J The Chlorella variabilis NC64A genome reveals adaptation to photosymbiosis, coevolution with viruses, and cryptic sex. 2010 Plant Cell 22(9):2943–2955
%
% 
%\noindent Darriba D, Taboada GL, Doallo R, Posada D. Prottest3: Fast selection of best-fit models of protein evolution. Bioinformatics, 27:1164-1165, 2011
%
% 
%\noindent Dolan, John, Mixotrophy in Ciliates: A Review of Chlorella Symbiosis and Chloroplast Retention, 1992,  Marine Microbial Food Webs,  6(2)
%
% 
%\noindent Edgar, R.C MUSCLE: multiple sequence alignment with high accuracy and high throughput, 2004, Nucleic Acids Res. 32(5):1792-1797 
%
% 
%\noindent Eddy, S.R., Accelerated Profile HMM Searches, 2011, PLoS Comp. Biol.
%
% 
%\noindent Emanuelsson, O., Nielsen, H., Brunak, S., von Heijne, G., Predicting subcellular localization of proteins based on their N-terminal amino acid sequence. J. Mol. Biol.,2000. 300: 1005-1016,
%
% 
%\noindent Fujishima, Masahiro,  Kodama, Yuuki, Endosymbionts in paramecium, 2012, European journal of protistology, 48,2
%
% 
%\noindent Gouy M., Guindon S. \& Gascuel O. (2010) SeaView version 4 : a multiplatform graphical user interface for sequence alignment and phylogenetic tree building. Molecular Biology and Evolution 27(2):221-224.
%
% 
%\noindent Guindon S., Dufayard J.F., Lefort V., Anisimova M., Hordijk W., Gascuel O. "New Algorithms and Methods to Estimate Maximum-Likelihood Phylogenies: Assessing the Performance of PhyML 3.0." Systematic Biology, 59(3):307-21, 2010.
%
% 
%\noindent Krogh A., Larsson B., von Heijne G., Sonnhammer EL., Predicting transmembrane protein topology with a hidden Markov model: application to complete genomes. J Mol Biol. 2001 Jan 19;305(3): 567-80.
%
% 
%\noindent Haas et al., De novo transcript sequence reconstruction from RNA-seq using the Trinity platform for reference generation and analysis, Nat. Protocols 2013 XXX
%
% 
%\noindent Harper, D.S., Jahn, C.L. Differential use of termination codons in ciliated protozoa 1989, PNAS, Vol. 86, pp. 3252-3256, XX
%
% 
%\noindent Horton, P., Park, K-J., Obayashi, T., Fujita, N., Harada, H., Adams-Collier, CJ., Nakai, K., WoLF PSORT: protein localization predictor Nucl. Acids Res. 2007 35 suppl 2: 585-587
%
% 
%\noindent A. Conesa, S. Götz, J. M. Garcia-Gomez, J. Terol, M. Talon and M. Robles. "Blast2GO: a universal tool for annotation, visualization and analysis in functional genomics research", Bioinformatics, Vol. 21, September, 2005, pp. 3674-3676.
%
% 
%\noindent Hoshina, R., Imamura, N,  Multiple Origins of the Symbioses in Paramecium bursaria 2008 Protist, vol. 159, 53-63
%
% 
%\noindent Hoshina, R., Imamura N., Origins of algal symbions in Paramecium Bursaria 2009, Microbiology Monographs 12 Endosymbtions in Paramecium (ed: Fujishima M.)
%
% 
%\noindent Hosoya, Hiroshi,  Kimura, Kouki, Matsuda, Seiji, Kitaura, Miyuki,  Takahashi, Tadao, Kosaka, Toshikazu, Symbiotic Algae-free Strains of The Green Paramecium Paramecium bursaria produced by the Herbicide Paraquat, 1995, Zoological Science, 12
%
% 
%\noindent Johnson, Matthew D,  Acquired phototrophy in ciliates: a review of cellular interactions and structural adaptations , 2011, The Journal of eukaryotic microbiology, 58(3)
%
% 
%\noindent Karakashian, S.J., Growth of Paramecium bursaria as Influenced by the Presence of Algal Symbionts, 1963,  Physiological Zoology, 36(1)
%
% 
%\noindent Karkashian, S.J.,  Karkashian, M.W., Evolution and Symbiosis in the Genus Chlorella and Related Algae, 1965, Evolution,  19(3)
%
% 
%\noindent Kato Y, Imamura, N. Metabolic Control Between the Symbiotic Chlorella and the Host Paramecium, 2009,  Microbiology Monographs 12,  Endosymbionts in Paramecium
%
% 
%\noindent Kodama, Yuuki, Nakahara, Miho, Fujishima, Masahiro,  Symbiotic alga Chlorella vulgaris of the ciliate Paramecium bursaria shows temporary resistance to host lysosomal enzymes during the early infection process, 2007, Protoplasma, 230(1,2)
%
% 
%\noindent Kodama, Yuuki, Fujishima, Masahiro, Infection of Paramecium bursaria by Symbiotic Chlorella Species, 2009, Microbiology Monographs 12,  Infection of Paramecium bursaria by Symbiotic Chlorella Species
%
% 
%\noindent Kodama, Y, Fujishima, M,  Induction of secondary symbiosis between the ciliate Paramecium and the green alga Chlorella 2010,  Current Research, Technology and Education Topics in Applied Microbiology and Microbial Biotechnology
%
% 
%\noindent Kodama, Yuuki, Inouye, Isao,  Fujishima, Masahiro, Symbiotic Chlorella vulgaris of the ciliate Paramecium bursaria plays an important role in maintaining perialgal vacuole membrane functions, 2011, Protist, 162(2)
%
% 
%\noindent Kadono T , Kawano T , Hosoya H , and Kosaka T (2004) Flow cytometric studies of the host-regu- lated cell cycle in algae symbiotic with green paramecium . Protoplasma 223 : 133 – 141
%
% 
%\noindent Kodama, Yuuki, Fujishima, Masahiro, Endosymbiosis of Chlorella species to the ciliate Paramecium bursaria alters the distribution of the host's trichocysts beneath the host cell cortex.2011, Protoplasma, 248(2)
%
% 
%\noindent Kodama, Yuuki,  Fujishima, Masahiro, Characteristics of the digestive vacuole membrane of the alga-bearing ciliate Paramecium bursaria , 2012  Protist 163(4)
%
% 
%\noindent Langmead B, Salzberg S. Fast gapped-read alignment with Bowtie 2. Nature Methods. 2012, 9:357-359.
%
% 
%\noindent Loefer,J. B., Bacteria free cultures of Paramecium bursaria and concentrations of the medium as a factor in growth. 1936 J. Exp. Zoo!., 72: 387
%
% 
%\noindent Mayer, K.M., Forney J.D., A Mutation in the Flanking 5'-TA-3' Dinucleotide Prevents Excision of an Internal Eliminated Sequence From the Paramecium tetraurelia Genome 1999 genetics vol. 151 no. 2 597-604
%
% 
%\noindent Miwa I., Regulation of Circadian Rhythms of Paramecium bursaria by Symbiotic Chlorella Species 2009, Microbiology Monographs 12, Endosymbions in Paramecium.
%
% 
%\noindent Meier, Renate, Wiessner, Wolfgang, Infection of algae-free Pammecium bursaria with symbiotic Chlorella sp. isolated from green paramecia, 1989, Journal of cell science, 93
%
% 
%\noindent{} Muscatine L Glycerol excretion by symbiotic algae from corals and Tridacna and its control by the host 1967  Science 156 : 516 – 519
%
% 
%\noindent Nishihara, N, Horiike, S,  Takahashi, T, Kosaka, T, Shigenaka, Y, Hosoya, H, Cloning and characterization of endosymbiotic algae isolated from Paramecium bursaria 1998, Protoplasma, 203
%
% 
%\noindent Parker, Raymond C, Symbiosis in Paramecium bursaria, 1926, Journal of Experimental Zoology,  46
%
% 
%\noindent Pérez MT , Dolan JR , Fukai E (1997) Planktonic oligotrich ciliates in the NW Mediterranean growth rates and consumption by copepods . Mar Ecol Prog Ser 155 : 89 – 101
%
% 
%\noindent Petersen, TN., Soren, B., von Heijne, G., Nielsen, H., SignalP 4.0: discriminating signal peptides from transmembrane regions. Nat Meth 2011; 8(10):785-786
%
% 
%\noindent Price, M.N., Dehal, P.S., and Arkin, A.P. FastTree 2 -- Approximately Maximum-Likelihood Trees for Large Alignments. 2010 PLoS ONE, 5(3):e9490.
%
% 
%\noindent Putt M Abundance, chlorophyll content and photosynthetic rates of clliates in the Nordic Seds during summer.1990  Deep-Sea Res 37:1713-1731
%
% 
%\noindent Raven JA Phagotrophy in phototrophs 1997 Limnol Oceanography 42:198–205
%
% 
%\noindent Reisser W  The metabolic interactions between Paramecium bursaria Ehrbg. and Chlorella spec. in the Paramecium bursaria -symbiosis III. The influence of different CO2 concentrations and of glucose on the photosynthetic and respiratory capacity of the symbiotic unit  1980. Arch Microbiol 125 : 291 – 293
%
% 
%\noindent Salvador Capella-Gutierrez; Jose M. Silla-Martinez; Toni Gabaldon. trimAl: a tool for automated alignment trimming in large-scale phylogenetic analyses. Bioinformatics 2009 25: 1972-1973.
%
% 
%\noindent M.H. Schulz, D.R. Zerbino, M. Vingron and Ewan Birney. Oases: Robust de novo RNA-seq assembly across the dynamic range of expression levels. Bioinformatics, 2012. DOI: 10.1093/bioinformatics/bts094.
%
% 
%\noindent Sonnhammer, EL, von Heijne, G., Krogh, A., A hidden Markov model for predicting transmembrane helices in protein sequences. Proc Int Conf Intell Syst Mol Biol 1998;6:175-82
%
% 
%\noindent Sommaruga, R., Sonntag, B., Photobiological Aspects of the Mutualistic Association Between Paramecium bursaria and Chlorella 2009, Microbiological Monographs 12, Endosymbionts in Paramecium
%
% 
%\noindent Summerer, Monika, Sonntag, Bettina,  Sommaruga, Ruben,  An experimental test of the symbiosis specificity between the ciliate Paramecium bursaria and strains of the unicellular green alga Chlorella., 2007,  Environmental microbiology, 9(8)
%
% 
%\noindent Summerer, Monika, Sonntag, Bettina, Hörtnagl, Paul, Sommaruga, Ruben, Symbiotic ciliates receive protection against UV damage from their algae: a test with Paramecium bursaria and Chlorella. 2009,  Protist, 160(2)
%
% 
%\noindent Taguchi, K, Hirota, S, Nakayama, H, Kunugihara, D, Mihara, Y - Optical Manipulation of Symbiotic Chlorella in Paramecium Bursaria Using a Fiber Axicon Microlens, 2012, Journal of Physics: Conference Series, 352
%
% 
%\noindent Takeda H , Sekiguchi T , Nunokawa S , Usuki I (1998) Species-specificity of Chlorella for estab- lishment of symbiotic association with Paramecium bursaria . Does infectivity depend upon sugar components of the cell wall? Eur J Protistol 34 : 133 – 137
%
% 
%\noindent Takahashi T , Shirai Y , Kosaka T , Hosoya H (2007) Arrest of cytoplasmic streaming induces algal proliferation in green paramecia . PLoS ONE 2 (12) : e1352 . doi: 10.1371/journal.pone.0001352
%
% 
%\noindent Van Etten JL , Meints HR , Kuczmarski D , Meints RH  Virus infection for culturable Chlorella-like algae and development of a plaque assay (1983). Science 219 : 994 – 996
%
% 
%\noindent Wootton, E. C., et al. Biochemical prey recognition by planktonic protozoa. 2007 , Environ. Microbiol. 9: 216 –222.
%
% 
%\noindent Weis, Dale S, Ayala, Alfred, Effect of Exposure Period and Algal Concentration on the Frequency of Infection of Aposymbiotic Ciliates by Symbiotic Algae from Paramecium bursaria, 1976,  Journal of protozoology, 26(2)
%
% 
%\noindent Yashchenko, Varvara V, Gavrilova, Olga V,  Rautian, Maria S, Jakobsen, Kjetill S, Association of Paramecium bursaria Chlorella viruses with Paramecium bursaria cells: ultrastructural studies. ,2012,  European journal of protistology, 48(2)
%
%
%


