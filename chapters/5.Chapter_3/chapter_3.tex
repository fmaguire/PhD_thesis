\graphicspath{{chapters/5.Chapter_3/figures}}

\chapter{Transporters}

\section{Introduction}

\section{Aims}

\section{Methods}

\subsection{Data Acquisition}

Sets of algal predicted proteins were acquired as follows:

\subsubsection{\textit{Chlorella variabilis} NC64A and \textit{Coccomyxa subellipsoidea} C-169} 
 
\textit{Coccomyxa subellipsoidea} C-169 genome project \citep{Blanc2012} version 2.0 
JGI annotated proteins (created 12-01-2014) were downloaded from JGI's
Phyotozome v10.3.1 \citep{Goodstein2012}. 
Similarly, the ``best'' annotated proteins from
version 1 of the \textit{Chlorella variabilis} NC64A genome project \citep{Blanc2010}
were downloaded from JGI's genome portal \citep{Grigoriev2011,Nordberg2014}

\subsubsection{\textit{Micractinium reisseri}}

Sequences were derived from endosymbiont binned
transcripts and annotations as described in \ref{chap:transcriptomics}.

\subsubsection{\textit{Chlorella variabilis 1 N}}

\(232.3M\) 100bp paired-end reads from \citep{Kodama2014}'s 
bulk RNAseq transcriptome of \textit{Paramecium bursaria} Yad1g (syngen
3, mating type 1) bearing \textit{Chlorella variabilis} 1N endosymbionts
were downloaded from the DNA Data Bank of Japan (DDBJ) \citep{Tateno2002,Kaminuma2011}
in Sequence Read Archive (SRA) format \citep{Leinonen2011,KodamaNRA2012b}. 

These reads were then converted to fastq using ``fastq-dump'' using the SRA Toolkit
\citep{NationalCenterforBiotechnologyInformation2011}.  Reads were then trimmed
for sequencing adapters using ILLUMINACLIP and SLIDINGWINDOW with a window size
of 4 and a minimum average quality of 30.













sequencing


DDBJ Sequence Read Archive (DRA) (accession number DRA000907, http://trace.ddbj.nig.ac.jp/DRASearch/submission?acc=DRA000907



\subsection{Transporter Identification}

\subsection{qPCR}


\section{Results}


\subsection{Kodama Assembly}



\subsubsection{Assembly}



\subsection{Transporter identification}

To identify transporters present in the translated protein dataset of the 
endosymbiont bins of the CCAP1660/12 and YADGN1 
\textit{Paramecium bursaria} transcriptome assemblies, as well as the Chlorella NC64A 
and Coccoymxa C-169 predicted proteomes the following process was used:

\begin{enumerate}
        \item TMHMM was used to predict TM domains within each protein
        \item Proteins were then filtered on those of which had 3 or more TM domains
        \item Those proteins were then searched using HMM and the PFAM HMM database
        \item Those hits were parsed for those which shared a PFAM domain with TCDB proteins
\end{enumerate}

\section{Discussion}

\section{Conclusion}


Network analysis
http://www.ncbi.nlm.nih.gov/pmc/articles/PMC3299011/pdf/fmicb-03-00085.pdf



%Genome assembly - spades macmanes trimming was worse than spades harash trimming (shorter contigs)
