The photosynthetic endosymbioses between \textit{Paramecium bursaria} and 
its green algal endosymbionts (\textit{Chlorella variabilis}, \textit{Chlorella vulgaris},
\textit{Micractinium reisseri} and \textit{Coccomyxa} sp.) 
have long been suggested to represent 
nascent endosymbiotic interactions as host and endosymbiont are believed
to be able to exist and reproduce separately.  
Understanding the molecular systems underpinning these relationships could therefore
provide model systems to understand the process of photosynthetic endosymbioses 
before molecular
co-dependence has become fixed.  

To this end, the  metatranscriptome of \textit{P. bursaria}-\textit{M. reisseri}
during lit and dark conditions was recovered using single cell methods. 
This necessitated the development of novel techniques to optimise the 
assembly and the post-assembly attribution of transcripts to their originating organism.
As such, this work represents the first \textit{de novo} single cell transcriptomic analysis
of a multimember eukaryotic system.  

In combination with a \textit{P. bursaria}-\textit{C. variabilis}
transcriptome and mass-spectrometry metabolomics data, this 
data was used to investigate metabolic function in endosymbioses. 
This identified potential roles for novel sugar, amino acid, 
and fatty acid interactions in the \textit{M. reisseri} endosymbiosis.
Additionally, \textit{P. bursaria} SW1 was discovered to form an obligate host of 
\textit{M. reisseri} SW1-ZK.


This work also reveals a potentially non-functional exogenous RNA incuded RNAi system in 
\textit{P. bursaria} likely related to the absence of a factor associated with uptake
of RNA from host vacuoles in both \textit{P. bursaria} transcriptomes and a partial
\textit{P. bursaria} single cell genome.  An analysis of the level of potential RNAi ``cross-talk''
collisions with the active host transcriptome 
suggest that the function of an exogenous RNA induced RNAi system in the presence
of a eukaryotic endosymbiont may be deleterious. 


Therefore, despite discovering several barriers to the utility of
these systems as general models for endosymbiotic evolution
there is still utility in their study.
The ``omic'' resources presented here offer an important 
resource in guiding further analysis.
