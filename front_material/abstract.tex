% the abstract 300 words

The photosynthetic endosymbioses between \textit{Paramecium bursaria} and 
its green algal endosymbionts have long been suggested to represent 
nascent endosymbiotic interactions as both partners are believed
to be able to exist and reproduce separately.

Using single cell genomics of \texit{P. bursaria}-\textit{Micractinium
reisseri} 


1. Not a facultative interaction in the case of \textit{P. bursaria}-\textit{M. reisseri} SW1-ZK
2. Mislabelled Coccomyxa culture in CCAP actually contains \textit{Micractinium}
3. MDA-based single cell metatranscriptomics is problematic in complex
eukaryotic systems, contamination from food bacteria.
4. \textit{M. reisseri} utilises different amino acids than previously studied
\textit{Paramecium}. 
5. There is a previously unidentified role for raffinose and arabinose in this relationship.
6. Key elements of the feeding dsRNA induced RNAi pathway are missing in \textit{P. bursaria}



\textit{Chlorella variabilis}, \textit{Chlorella vulgaris} and \textit{Coccomyxa} have long 
been suggested to represent 
Understanding the molecular systems underpinning these relationships would provide
model systems to understand the process of photosynthetic endosymbioses before molecular
co-dependence has become fixed.  Using a combination of single cell transcriptomics and genomics
as well as targeted and untargeted metabolomics at day and night.
