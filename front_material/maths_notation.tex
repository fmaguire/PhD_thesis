Throughout this thesis several standard mathematical conventions have been used 
(inline with the ISO80000-2 regulations for technical writing \citet{ISO2009}).

Specifically, matrices are denoted by boldface italic capital letters 
(\(\mathbfit{A}, \mathbfit{B}, \mathbfit{X}\)) 
and their elements by standard italic lower case letters indexed 
using subscripts
(i.e. \(a_{ij}\) is the element \(i,j\) of the matrix
\(\mathbfit{A}\)). 
Vectors are denoted by an arrow above standard italic lowercase letters 
(\(\vec{x}, \vec{y}, \vec{z}\)) with their elements 
indicated in the same manner as matrix elements (\(x_{i}\) is the 
\(i\)th element of the vector \(\vec{x}\)). 
Tensors of third and higher orders are represented by boldface italic
sans serif capitals (\(\mathsfbfit{T}, \mathsfbfit{L}\)).
Finally, scalars are denoted by standard lowercase italic 
elements (\(k, n, p\)).

Variables which may be a tensor of any order 
are indicated using a lower case sanserif letter (\(\mathsf{a}, \mathsf{b}\)).

Norms are used and denoted in standard linear algebra fashion. 
The norm of a scalar is equivalent to its absolute value (\(\norm{n} = \abs{n}\)).
Unless stated otherwise all matrix and vector norms 
(\(\norm{\mathbfit{A}}, \norm{\vec{y}}\)) are the Euclidean/\(\ell-2\) norms.
Otherwise the norm is indicated by a subset number in place of \(p\):
\(\norm{\vec{x}}_{p} = (\sum^{n}_{i=1} \abs{x_{i}}^{p})^{\frac{1}{p}}\).



